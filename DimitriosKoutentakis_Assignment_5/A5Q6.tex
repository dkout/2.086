\documentclass[14pt]{article}
\usepackage{graphicx}
\usepackage{amsmath}
\usepackage{pgfplots}
\usepackage{geometry} 
\geometry{a4paper}
\pgfplotsset{compat=1.12}


\title{2.086 -A5Q6 }
\author{Dimitrios Koutentakis}
\date{}

\begin{document}
\maketitle
%\vspace{1cm}
%\vspace{1cm}

\subsection*{Which approximation, finite-difference (based on interpolation) or best-fit, is most accurate?}
%\vspace{1cm}


The approximation that provides the most accuracy is the best-fit approximation. That is because the best-fit approximation is designed to pass as close as possible from all of our data-points. On the other hand, the quadratic interpolation is designed to only accomodate three points. We anticipate that in the plot of Height-vs-Time the quadratic approximation line passes perfectly through the three datapoints we used but it falls away from the points further from $t_0$, while the best-fit solution should pass very close to all of the points (not necessarily through any of them).Obviously the more datapoints our approximation accomodates, the less significant the error becomes. In the quadratic interpolation, the use of three datapoints means that the random error will have a bigger effect in the approximation than when using the best-fit solution (where we use five data points).\\

\subsection*{Plot is attached below}

\begin{figure}[b!]
\centering 
\newlength\figureheight
 \newlength\figurewidth 
\setlength\figureheight{6cm}
 \setlength\figurewidth{12.5cm} 
% This file was created by matlab2tikz.
%
%The latest updates can be retrieved from
%  http://www.mathworks.com/matlabcentral/fileexchange/22022-matlab2tikz-matlab2tikz
%where you can also make suggestions and rate matlab2tikz.
%
\definecolor{mycolor1}{rgb}{0.00000,0.44700,0.74100}%
\definecolor{mycolor2}{rgb}{0.85000,0.32500,0.09800}%
\definecolor{mycolor3}{rgb}{0.92900,0.69400,0.12500}%
%
\begin{tikzpicture}

\begin{axis}[%
width=0.951\figurewidth,
height=\figureheight,
at={(0\figurewidth,0\figureheight)},
scale only axis,
xmin=0,
xmax=0.14,
xlabel={Time [s]},
ymin=1.84,
ymax=2.02,
ylabel={Distance from ground (height) [m]},
axis background/.style={fill=white},
title style={font=\bfseries},
title={Height vs Time (Data and Estimations)},
axis x line*=bottom,
axis y line*=left,
legend style={legend cell align=left,align=left,draw=white!15!black}
]
\addplot[only marks,mark=o,mark options={},mark size=1.5000pt,color=mycolor1] plot table[row sep=crcr]{%
0	1.99965742574257\\
0.0333333333333333	1.97928712871287\\
0.0666666666666667	1.94612753623188\\
0.1	1.90857971014493\\
0.133333333333333	1.85630724637681\\
};
\addlegendentry{Data Points};

\addplot [color=mycolor2,solid]
  table[row sep=crcr]{%
0	1.99973156960416\\
0.0001	1.99968275420468\\
0.0002	1.99963385112904\\
0.0003	1.99958486037724\\
0.0004	1.99953578194928\\
0.0005	1.99948661584517\\
0.0006	1.9994373620649\\
0.0007	1.99938802060847\\
0.0008	1.99933859147588\\
0.0009	1.99928907466714\\
0.001	1.99923947018223\\
0.0011	1.99918977802117\\
0.0012	1.99913999818396\\
0.0013	1.99909013067058\\
0.0014	1.99904017548105\\
0.0015	1.99899013261536\\
0.0016	1.99894000207351\\
0.0017	1.9988897838555\\
0.0018	1.99883947796134\\
0.0019	1.99878908439102\\
0.002	1.99873860314454\\
0.0021	1.9986880342219\\
0.0022	1.99863737762311\\
0.0023	1.99858663334815\\
0.0024	1.99853580139704\\
0.0025	1.99848488176978\\
0.0026	1.99843387446635\\
0.0027	1.99838277948677\\
0.0028	1.99833159683103\\
0.0029	1.99828032649913\\
0.003	1.99822896849107\\
0.0031	1.99817752280686\\
0.0032	1.99812598944649\\
0.0033	1.99807436840996\\
0.0034	1.99802265969727\\
0.0035	1.99797086330843\\
0.0036	1.99791897924342\\
0.0037	1.99786700750226\\
0.0038	1.99781494808495\\
0.0039	1.99776280099147\\
0.004	1.99771056622184\\
0.0041	1.99765824377605\\
0.0042	1.9976058336541\\
0.0043	1.99755333585599\\
0.0044	1.99750075038173\\
0.0045	1.99744807723131\\
0.0046	1.99739531640473\\
0.0047	1.99734246790199\\
0.0048	1.9972895317231\\
0.0049	1.99723650786805\\
0.005	1.99718339633684\\
0.0051	1.99713019712947\\
0.0052	1.99707691024595\\
0.0053	1.99702353568626\\
0.0054	1.99697007345042\\
0.0055	1.99691652353842\\
0.0056	1.99686288595027\\
0.0057	1.99680916068596\\
0.0058	1.99675534774548\\
0.0059	1.99670144712886\\
0.006	1.99664745883607\\
0.0061	1.99659338286713\\
0.0062	1.99653921922202\\
0.0063	1.99648496790076\\
0.0064	1.99643062890335\\
0.0065	1.99637620222977\\
0.0066	1.99632168788004\\
0.0067	1.99626708585415\\
0.0068	1.9962123961521\\
0.0069	1.9961576187739\\
0.007	1.99610275371953\\
0.0071	1.99604780098901\\
0.0072	1.99599276058233\\
0.0073	1.9959376324995\\
0.0074	1.9958824167405\\
0.0075	1.99582711330535\\
0.0076	1.99577172219404\\
0.0077	1.99571624340658\\
0.0078	1.99566067694295\\
0.0079	1.99560502280317\\
0.008	1.99554928098723\\
0.0081	1.99549345149513\\
0.0082	1.99543753432688\\
0.0083	1.99538152948246\\
0.0084	1.99532543696189\\
0.0085	1.99526925676516\\
0.0086	1.99521298889228\\
0.0087	1.99515663334323\\
0.0088	1.99510019011803\\
0.0089	1.99504365921667\\
0.009	1.99498704063916\\
0.0091	1.99493033438548\\
0.0092	1.99487354045565\\
0.0093	1.99481665884966\\
0.0094	1.99475968956751\\
0.0095	1.99470263260921\\
0.0096	1.99464548797474\\
0.0097	1.99458825566412\\
0.0098	1.99453093567735\\
0.0099	1.99447352801441\\
0.01	1.99441603267532\\
0.0101	1.99435844966007\\
0.0102	1.99430077896866\\
0.0103	1.99424302060109\\
0.0104	1.99418517455737\\
0.0105	1.99412724083748\\
0.0106	1.99406921944144\\
0.0107	1.99401111036925\\
0.0108	1.99395291362089\\
0.0109	1.99389462919638\\
0.011	1.99383625709571\\
0.0111	1.99377779731888\\
0.0112	1.9937192498659\\
0.0113	1.99366061473675\\
0.0114	1.99360189193145\\
0.0115	1.99354308144999\\
0.0116	1.99348418329238\\
0.0117	1.9934251974586\\
0.0118	1.99336612394867\\
0.0119	1.99330696276258\\
0.012	1.99324771390033\\
0.0121	1.99318837736193\\
0.0122	1.99312895314737\\
0.0123	1.99306944125665\\
0.0124	1.99300984168977\\
0.0125	1.99295015444673\\
0.0126	1.99289037952754\\
0.0127	1.99283051693219\\
0.0128	1.99277056666068\\
0.0129	1.99271052871301\\
0.013	1.99265040308919\\
0.0131	1.99259018978921\\
0.0132	1.99252988881307\\
0.0133	1.99246950016077\\
0.0134	1.99240902383232\\
0.0135	1.99234845982771\\
0.0136	1.99228780814694\\
0.0137	1.99222706879001\\
0.0138	1.99216624175692\\
0.0139	1.99210532704768\\
0.014	1.99204432466228\\
0.0141	1.99198323460072\\
0.0142	1.991922056863\\
0.0143	1.99186079144913\\
0.0144	1.9917994383591\\
0.0145	1.99173799759291\\
0.0146	1.99167646915056\\
0.0147	1.99161485303206\\
0.0148	1.9915531492374\\
0.0149	1.99149135776658\\
0.015	1.9914294786196\\
0.0151	1.99136751179647\\
0.0152	1.99130545729717\\
0.0153	1.99124331512172\\
0.0154	1.99118108527011\\
0.0155	1.99111876774235\\
0.0156	1.99105636253842\\
0.0157	1.99099386965834\\
0.0158	1.9909312891021\\
0.0159	1.99086862086971\\
0.016	1.99080586496115\\
0.0161	1.99074302137644\\
0.0162	1.99068009011557\\
0.0163	1.99061707117855\\
0.0164	1.99055396456536\\
0.0165	1.99049077027602\\
0.0166	1.99042748831052\\
0.0167	1.99036411866886\\
0.0168	1.99030066135104\\
0.0169	1.99023711635707\\
0.017	1.99017348368694\\
0.0171	1.99010976334065\\
0.0172	1.9900459553182\\
0.0173	1.9899820596196\\
0.0174	1.98991807624484\\
0.0175	1.98985400519392\\
0.0176	1.98978984646684\\
0.0177	1.98972560006361\\
0.0178	1.98966126598422\\
0.0179	1.98959684422867\\
0.018	1.98953233479696\\
0.0181	1.98946773768909\\
0.0182	1.98940305290507\\
0.0183	1.98933828044489\\
0.0184	1.98927342030855\\
0.0185	1.98920847249605\\
0.0186	1.9891434370074\\
0.0187	1.98907831384259\\
0.0188	1.98901310300162\\
0.0189	1.98894780448449\\
0.019	1.98888241829121\\
0.0191	1.98881694442177\\
0.0192	1.98875138287617\\
0.0193	1.98868573365441\\
0.0194	1.98861999675649\\
0.0195	1.98855417218242\\
0.0196	1.98848825993219\\
0.0197	1.9884222600058\\
0.0198	1.98835617240325\\
0.0199	1.98828999712455\\
0.02	1.98822373416969\\
0.0201	1.98815738353867\\
0.0202	1.98809094523149\\
0.0203	1.98802441924816\\
0.0204	1.98795780558867\\
0.0205	1.98789110425302\\
0.0206	1.98782431524121\\
0.0207	1.98775743855325\\
0.0208	1.98769047418912\\
0.0209	1.98762342214884\\
0.021	1.9875562824324\\
0.0211	1.98748905503981\\
0.0212	1.98742173997106\\
0.0213	1.98735433722614\\
0.0214	1.98728684680508\\
0.0215	1.98721926870785\\
0.0216	1.98715160293446\\
0.0217	1.98708384948492\\
0.0218	1.98701600835922\\
0.0219	1.98694807955737\\
0.022	1.98688006307935\\
0.0221	1.98681195892518\\
0.0222	1.98674376709485\\
0.0223	1.98667548758836\\
0.0224	1.98660712040571\\
0.0225	1.98653866554691\\
0.0226	1.98647012301195\\
0.0227	1.98640149280083\\
0.0228	1.98633277491356\\
0.0229	1.98626396935012\\
0.023	1.98619507611053\\
0.0231	1.98612609519478\\
0.0232	1.98605702660287\\
0.0233	1.98598787033481\\
0.0234	1.98591862639059\\
0.0235	1.98584929477021\\
0.0236	1.98577987547367\\
0.0237	1.98571036850097\\
0.0238	1.98564077385212\\
0.0239	1.98557109152711\\
0.024	1.98550132152594\\
0.0241	1.98543146384862\\
0.0242	1.98536151849513\\
0.0243	1.98529148546549\\
0.0244	1.98522136475969\\
0.0245	1.98515115637773\\
0.0246	1.98508086031962\\
0.0247	1.98501047658535\\
0.0248	1.98494000517492\\
0.0249	1.98486944608833\\
0.025	1.98479879932558\\
0.0251	1.98472806488668\\
0.0252	1.98465724277162\\
0.0253	1.9845863329804\\
0.0254	1.98451533551303\\
0.0255	1.98444425036949\\
0.0256	1.9843730775498\\
0.0257	1.98430181705395\\
0.0258	1.98423046888195\\
0.0259	1.98415903303378\\
0.026	1.98408750950946\\
0.0261	1.98401589830898\\
0.0262	1.98394419943234\\
0.0263	1.98387241287955\\
0.0264	1.9838005386506\\
0.0265	1.98372857674548\\
0.0266	1.98365652716422\\
0.0267	1.98358438990679\\
0.0268	1.98351216497321\\
0.0269	1.98343985236347\\
0.027	1.98336745207757\\
0.0271	1.98329496411551\\
0.0272	1.9832223884773\\
0.0273	1.98314972516293\\
0.0274	1.9830769741724\\
0.0275	1.98300413550571\\
0.0276	1.98293120916286\\
0.0277	1.98285819514386\\
0.0278	1.9827850934487\\
0.0279	1.98271190407738\\
0.028	1.98263862702991\\
0.0281	1.98256526230627\\
0.0282	1.98249180990648\\
0.0283	1.98241826983053\\
0.0284	1.98234464207843\\
0.0285	1.98227092665016\\
0.0286	1.98219712354574\\
0.0287	1.98212323276516\\
0.0288	1.98204925430843\\
0.0289	1.98197518817553\\
0.029	1.98190103436648\\
0.0291	1.98182679288127\\
0.0292	1.9817524637199\\
0.0293	1.98167804688238\\
0.0294	1.98160354236869\\
0.0295	1.98152895017885\\
0.0296	1.98145427031285\\
0.0297	1.9813795027707\\
0.0298	1.98130464755238\\
0.0299	1.98122970465791\\
0.03	1.98115467408728\\
0.0301	1.9810795558405\\
0.0302	1.98100434991755\\
0.0303	1.98092905631845\\
0.0304	1.98085367504319\\
0.0305	1.98077820609177\\
0.0306	1.9807026494642\\
0.0307	1.98062700516047\\
0.0308	1.98055127318057\\
0.0309	1.98047545352453\\
0.031	1.98039954619232\\
0.0311	1.98032355118396\\
0.0312	1.98024746849944\\
0.0313	1.98017129813876\\
0.0314	1.98009504010192\\
0.0315	1.98001869438893\\
0.0316	1.97994226099977\\
0.0317	1.97986573993446\\
0.0318	1.979789131193\\
0.0319	1.97971243477537\\
0.032	1.97963565068159\\
0.0321	1.97955877891165\\
0.0322	1.97948181946555\\
0.0323	1.9794047723433\\
0.0324	1.97932763754488\\
0.0325	1.97925041507031\\
0.0326	1.97917310491958\\
0.0327	1.9790957070927\\
0.0328	1.97901822158965\\
0.0329	1.97894064841045\\
0.033	1.97886298755509\\
0.0331	1.97878523902357\\
0.0332	1.9787074028159\\
0.0333	1.97862947893207\\
0.0334	1.97855146737208\\
0.0335	1.97847336813593\\
0.0336	1.97839518122362\\
0.0337	1.97831690663516\\
0.0338	1.97823854437054\\
0.0339	1.97816009442976\\
0.034	1.97808155681282\\
0.0341	1.97800293151973\\
0.0342	1.97792421855048\\
0.0343	1.97784541790507\\
0.0344	1.9777665295835\\
0.0345	1.97768755358578\\
0.0346	1.9776084899119\\
0.0347	1.97752933856186\\
0.0348	1.97745009953566\\
0.0349	1.9773707728333\\
0.035	1.97729135845479\\
0.0351	1.97721185640012\\
0.0352	1.97713226666929\\
0.0353	1.9770525892623\\
0.0354	1.97697282417916\\
0.0355	1.97689297141986\\
0.0356	1.9768130309844\\
0.0357	1.97673300287278\\
0.0358	1.97665288708501\\
0.0359	1.97657268362108\\
0.036	1.97649239248099\\
0.0361	1.97641201366474\\
0.0362	1.97633154717233\\
0.0363	1.97625099300377\\
0.0364	1.97617035115905\\
0.0365	1.97608962163817\\
0.0366	1.97600880444114\\
0.0367	1.97592789956794\\
0.0368	1.97584690701859\\
0.0369	1.97576582679308\\
0.037	1.97568465889142\\
0.0371	1.97560340331359\\
0.0372	1.97552206005961\\
0.0373	1.97544062912947\\
0.0374	1.97535911052317\\
0.0375	1.97527750424072\\
0.0376	1.97519581028211\\
0.0377	1.97511402864734\\
0.0378	1.97503215933641\\
0.0379	1.97495020234932\\
0.038	1.97486815768608\\
0.0381	1.97478602534668\\
0.0382	1.97470380533112\\
0.0383	1.9746214976394\\
0.0384	1.97453910227153\\
0.0385	1.9744566192275\\
0.0386	1.97437404850731\\
0.0387	1.97429139011096\\
0.0388	1.97420864403846\\
0.0389	1.97412581028979\\
0.039	1.97404288886497\\
0.0391	1.973959879764\\
0.0392	1.97387678298686\\
0.0393	1.97379359853357\\
0.0394	1.97371032640412\\
0.0395	1.97362696659851\\
0.0396	1.97354351911674\\
0.0397	1.97345998395882\\
0.0398	1.97337636112474\\
0.0399	1.9732926506145\\
0.04	1.9732088524281\\
0.0401	1.97312496656555\\
0.0402	1.97304099302683\\
0.0403	1.97295693181196\\
0.0404	1.97287278292094\\
0.0405	1.97278854635375\\
0.0406	1.97270422211041\\
0.0407	1.97261981019091\\
0.0408	1.97253531059525\\
0.0409	1.97245072332343\\
0.041	1.97236604837546\\
0.0411	1.97228128575133\\
0.0412	1.97219643545104\\
0.0413	1.97211149747459\\
0.0414	1.97202647182199\\
0.0415	1.97194135849322\\
0.0416	1.9718561574883\\
0.0417	1.97177086880723\\
0.0418	1.97168549244999\\
0.0419	1.9716000284166\\
0.042	1.97151447670705\\
0.0421	1.97142883732134\\
0.0422	1.97134311025948\\
0.0423	1.97125729552145\\
0.0424	1.97117139310727\\
0.0425	1.97108540301693\\
0.0426	1.97099932525044\\
0.0427	1.97091315980778\\
0.0428	1.97082690668897\\
0.0429	1.970740565894\\
0.043	1.97065413742287\\
0.0431	1.97056762127559\\
0.0432	1.97048101745215\\
0.0433	1.97039432595254\\
0.0434	1.97030754677679\\
0.0435	1.97022067992487\\
0.0436	1.9701337253968\\
0.0437	1.97004668319257\\
0.0438	1.96995955331218\\
0.0439	1.96987233575563\\
0.044	1.96978503052293\\
0.0441	1.96969763761407\\
0.0442	1.96961015702905\\
0.0443	1.96952258876787\\
0.0444	1.96943493283053\\
0.0445	1.96934718921704\\
0.0446	1.96925935792739\\
0.0447	1.96917143896158\\
0.0448	1.96908343231962\\
0.0449	1.9689953380015\\
0.045	1.96890715600722\\
0.0451	1.96881888633678\\
0.0452	1.96873052899018\\
0.0453	1.96864208396743\\
0.0454	1.96855355126852\\
0.0455	1.96846493089345\\
0.0456	1.96837622284222\\
0.0457	1.96828742711484\\
0.0458	1.96819854371129\\
0.0459	1.96810957263159\\
0.046	1.96802051387574\\
0.0461	1.96793136744372\\
0.0462	1.96784213333555\\
0.0463	1.96775281155122\\
0.0464	1.96766340209073\\
0.0465	1.96757390495408\\
0.0466	1.96748432014128\\
0.0467	1.96739464765232\\
0.0468	1.9673048874872\\
0.0469	1.96721503964592\\
0.047	1.96712510412849\\
0.0471	1.9670350809349\\
0.0472	1.96694497006515\\
0.0473	1.96685477151924\\
0.0474	1.96676448529717\\
0.0475	1.96667411139895\\
0.0476	1.96658364982457\\
0.0477	1.96649310057403\\
0.0478	1.96640246364734\\
0.0479	1.96631173904448\\
0.048	1.96622092676547\\
0.0481	1.9661300268103\\
0.0482	1.96603903917898\\
0.0483	1.96594796387149\\
0.0484	1.96585680088785\\
0.0485	1.96576555022805\\
0.0486	1.96567421189209\\
0.0487	1.96558278587998\\
0.0488	1.96549127219171\\
0.0489	1.96539967082728\\
0.049	1.96530798178669\\
0.0491	1.96521620506994\\
0.0492	1.96512434067704\\
0.0493	1.96503238860798\\
0.0494	1.96494034886276\\
0.0495	1.96484822144138\\
0.0496	1.96475600634385\\
0.0497	1.96466370357016\\
0.0498	1.96457131312031\\
0.0499	1.9644788349943\\
0.05	1.96438626919214\\
0.0501	1.96429361571381\\
0.0502	1.96420087455933\\
0.0503	1.9641080457287\\
0.0504	1.9640151292219\\
0.0505	1.96392212503895\\
0.0506	1.96382903317984\\
0.0507	1.96373585364457\\
0.0508	1.96364258643314\\
0.0509	1.96354923154556\\
0.051	1.96345578898182\\
0.0511	1.96336225874192\\
0.0512	1.96326864082586\\
0.0513	1.96317493523365\\
0.0514	1.96308114196527\\
0.0515	1.96298726102074\\
0.0516	1.96289329240006\\
0.0517	1.96279923610321\\
0.0518	1.96270509213021\\
0.0519	1.96261086048105\\
0.052	1.96251654115573\\
0.0521	1.96242213415425\\
0.0522	1.96232763947662\\
0.0523	1.96223305712283\\
0.0524	1.96213838709288\\
0.0525	1.96204362938677\\
0.0526	1.96194878400451\\
0.0527	1.96185385094609\\
0.0528	1.96175883021151\\
0.0529	1.96166372180077\\
0.053	1.96156852571388\\
0.0531	1.96147324195082\\
0.0532	1.96137787051161\\
0.0533	1.96128241139624\\
0.0534	1.96118686460472\\
0.0535	1.96109123013704\\
0.0536	1.96099550799319\\
0.0537	1.9608996981732\\
0.0538	1.96080380067704\\
0.0539	1.96070781550472\\
0.054	1.96061174265625\\
0.0541	1.96051558213162\\
0.0542	1.96041933393084\\
0.0543	1.96032299805389\\
0.0544	1.96022657450079\\
0.0545	1.96013006327153\\
0.0546	1.96003346436611\\
0.0547	1.95993677778454\\
0.0548	1.9598400035268\\
0.0549	1.95974314159291\\
0.055	1.95964619198286\\
0.0551	1.95954915469666\\
0.0552	1.95945202973429\\
0.0553	1.95935481709577\\
0.0554	1.95925751678109\\
0.0555	1.95916012879025\\
0.0556	1.95906265312326\\
0.0557	1.95896508978011\\
0.0558	1.9588674387608\\
0.0559	1.95876970006533\\
0.056	1.9586718736937\\
0.0561	1.95857395964592\\
0.0562	1.95847595792198\\
0.0563	1.95837786852188\\
0.0564	1.95827969144563\\
0.0565	1.95818142669321\\
0.0566	1.95808307426464\\
0.0567	1.95798463415991\\
0.0568	1.95788610637903\\
0.0569	1.95778749092198\\
0.057	1.95768878778878\\
0.0571	1.95758999697942\\
0.0572	1.9574911184939\\
0.0573	1.95739215233223\\
0.0574	1.95729309849439\\
0.0575	1.9571939569804\\
0.0576	1.95709472779025\\
0.0577	1.95699541092395\\
0.0578	1.95689600638149\\
0.0579	1.95679651416286\\
0.058	1.95669693426809\\
0.0581	1.95659726669715\\
0.0582	1.95649751145005\\
0.0583	1.9563976685268\\
0.0584	1.95629773792739\\
0.0585	1.95619771965183\\
0.0586	1.9560976137001\\
0.0587	1.95599742007222\\
0.0588	1.95589713876818\\
0.0589	1.95579676978798\\
0.059	1.95569631313162\\
0.0591	1.95559576879911\\
0.0592	1.95549513679044\\
0.0593	1.95539441710561\\
0.0594	1.95529360974462\\
0.0595	1.95519271470748\\
0.0596	1.95509173199418\\
0.0597	1.95499066160472\\
0.0598	1.9548895035391\\
0.0599	1.95478825779733\\
0.06	1.95468692437939\\
0.0601	1.9545855032853\\
0.0602	1.95448399451506\\
0.0603	1.95438239806865\\
0.0604	1.95428071394609\\
0.0605	1.95417894214737\\
0.0606	1.95407708267249\\
0.0607	1.95397513552145\\
0.0608	1.95387310069426\\
0.0609	1.95377097819091\\
0.061	1.9536687680114\\
0.0611	1.95356647015573\\
0.0612	1.95346408462391\\
0.0613	1.95336161141592\\
0.0614	1.95325905053178\\
0.0615	1.95315640197149\\
0.0616	1.95305366573503\\
0.0617	1.95295084182242\\
0.0618	1.95284793023365\\
0.0619	1.95274493096872\\
0.062	1.95264184402763\\
0.0621	1.95253866941039\\
0.0622	1.95243540711699\\
0.0623	1.95233205714743\\
0.0624	1.95222861950171\\
0.0625	1.95212509417984\\
0.0626	1.9520214811818\\
0.0627	1.95191778050762\\
0.0628	1.95181399215727\\
0.0629	1.95171011613076\\
0.063	1.9516061524281\\
0.0631	1.95150210104928\\
0.0632	1.9513979619943\\
0.0633	1.95129373526317\\
0.0634	1.95118942085587\\
0.0635	1.95108501877242\\
0.0636	1.95098052901281\\
0.0637	1.95087595157705\\
0.0638	1.95077128646512\\
0.0639	1.95066653367704\\
0.064	1.9505616932128\\
0.0641	1.9504567650724\\
0.0642	1.95035174925585\\
0.0643	1.95024664576313\\
0.0644	1.95014145459426\\
0.0645	1.95003617574924\\
0.0646	1.94993080922805\\
0.0647	1.94982535503071\\
0.0648	1.94971981315721\\
0.0649	1.94961418360755\\
0.065	1.94950846638173\\
0.0651	1.94940266147976\\
0.0652	1.94929676890163\\
0.0653	1.94919078864734\\
0.0654	1.94908472071689\\
0.0655	1.94897856511028\\
0.0656	1.94887232182752\\
0.0657	1.9487659908686\\
0.0658	1.94865957223352\\
0.0659	1.94855306592229\\
0.066	1.9484464719349\\
0.0661	1.94833979027134\\
0.0662	1.94823302093164\\
0.0663	1.94812616391577\\
0.0664	1.94801921922375\\
0.0665	1.94791218685556\\
0.0666	1.94780506681123\\
0.0667	1.94769785909073\\
0.0668	1.94759056369407\\
0.0669	1.94748318062126\\
0.067	1.94737570987229\\
0.0671	1.94726815144716\\
0.0672	1.94716050534588\\
0.0673	1.94705277156844\\
0.0674	1.94694495011483\\
0.0675	1.94683704098508\\
0.0676	1.94672904417916\\
0.0677	1.94662095969709\\
0.0678	1.94651278753886\\
0.0679	1.94640452770447\\
0.068	1.94629618019392\\
0.0681	1.94618774500722\\
0.0682	1.94607922214435\\
0.0683	1.94597061160533\\
0.0684	1.94586191339016\\
0.0685	1.94575312749882\\
0.0686	1.94564425393133\\
0.0687	1.94553529268768\\
0.0688	1.94542624376787\\
0.0689	1.9453171071719\\
0.069	1.94520788289978\\
0.0691	1.9450985709515\\
0.0692	1.94498917132706\\
0.0693	1.94487968402646\\
0.0694	1.94477010904971\\
0.0695	1.9446604463968\\
0.0696	1.94455069606773\\
0.0697	1.9444408580625\\
0.0698	1.94433093238112\\
0.0699	1.94422091902357\\
0.07	1.94411081798987\\
0.0701	1.94400062928002\\
0.0702	1.943890352894\\
0.0703	1.94377998883183\\
0.0704	1.9436695370935\\
0.0705	1.94355899767901\\
0.0706	1.94344837058836\\
0.0707	1.94333765582156\\
0.0708	1.94322685337859\\
0.0709	1.94311596325948\\
0.071	1.9430049854642\\
0.0711	1.94289391999276\\
0.0712	1.94278276684517\\
0.0713	1.94267152602142\\
0.0714	1.94256019752151\\
0.0715	1.94244878134545\\
0.0716	1.94233727749323\\
0.0717	1.94222568596484\\
0.0718	1.94211400676031\\
0.0719	1.94200223987961\\
0.072	1.94189038532276\\
0.0721	1.94177844308974\\
0.0722	1.94166641318058\\
0.0723	1.94155429559525\\
0.0724	1.94144209033376\\
0.0725	1.94132979739612\\
0.0726	1.94121741678232\\
0.0727	1.94110494849236\\
0.0728	1.94099239252625\\
0.0729	1.94087974888398\\
0.073	1.94076701756555\\
0.0731	1.94065419857096\\
0.0732	1.94054129190021\\
0.0733	1.94042829755331\\
0.0734	1.94031521553025\\
0.0735	1.94020204583103\\
0.0736	1.94008878845565\\
0.0737	1.93997544340412\\
0.0738	1.93986201067642\\
0.0739	1.93974849027257\\
0.074	1.93963488219257\\
0.0741	1.9395211864364\\
0.0742	1.93940740300408\\
0.0743	1.9392935318956\\
0.0744	1.93917957311096\\
0.0745	1.93906552665016\\
0.0746	1.93895139251321\\
0.0747	1.9388371707001\\
0.0748	1.93872286121083\\
0.0749	1.93860846404541\\
0.075	1.93849397920382\\
0.0751	1.93837940668608\\
0.0752	1.93826474649218\\
0.0753	1.93814999862212\\
0.0754	1.93803516307591\\
0.0755	1.93792023985353\\
0.0756	1.937805228955\\
0.0757	1.93769013038032\\
0.0758	1.93757494412947\\
0.0759	1.93745967020247\\
0.076	1.93734430859931\\
0.0761	1.93722885931999\\
0.0762	1.93711332236451\\
0.0763	1.93699769773288\\
0.0764	1.93688198542509\\
0.0765	1.93676618544114\\
0.0766	1.93665029778103\\
0.0767	1.93653432244477\\
0.0768	1.93641825943234\\
0.0769	1.93630210874376\\
0.077	1.93618587037903\\
0.0771	1.93606954433813\\
0.0772	1.93595313062108\\
0.0773	1.93583662922787\\
0.0774	1.9357200401585\\
0.0775	1.93560336341297\\
0.0776	1.93548659899129\\
0.0777	1.93536974689345\\
0.0778	1.93525280711945\\
0.0779	1.93513577966929\\
0.078	1.93501866454298\\
0.0781	1.9349014617405\\
0.0782	1.93478417126187\\
0.0783	1.93466679310709\\
0.0784	1.93454932727614\\
0.0785	1.93443177376904\\
0.0786	1.93431413258578\\
0.0787	1.93419640372636\\
0.0788	1.93407858719078\\
0.0789	1.93396068297905\\
0.079	1.93384269109116\\
0.0791	1.93372461152711\\
0.0792	1.9336064442869\\
0.0793	1.93348818937054\\
0.0794	1.93336984677802\\
0.0795	1.93325141650934\\
0.0796	1.9331328985645\\
0.0797	1.9330142929435\\
0.0798	1.93289559964635\\
0.0799	1.93277681867304\\
0.08	1.93265795002357\\
0.0801	1.93253899369795\\
0.0802	1.93241994969616\\
0.0803	1.93230081801822\\
0.0804	1.93218159866412\\
0.0805	1.93206229163387\\
0.0806	1.93194289692745\\
0.0807	1.93182341454488\\
0.0808	1.93170384448615\\
0.0809	1.93158418675127\\
0.081	1.93146444134022\\
0.0811	1.93134460825302\\
0.0812	1.93122468748966\\
0.0813	1.93110467905014\\
0.0814	1.93098458293446\\
0.0815	1.93086439914263\\
0.0816	1.93074412767464\\
0.0817	1.93062376853049\\
0.0818	1.93050332171019\\
0.0819	1.93038278721372\\
0.082	1.9302621650411\\
0.0821	1.93014145519232\\
0.0822	1.93002065766738\\
0.0823	1.92989977246629\\
0.0824	1.92977879958904\\
0.0825	1.92965773903563\\
0.0826	1.92953659080606\\
0.0827	1.92941535490033\\
0.0828	1.92929403131845\\
0.0829	1.92917262006041\\
0.083	1.92905112112621\\
0.0831	1.92892953451586\\
0.0832	1.92880786022934\\
0.0833	1.92868609826667\\
0.0834	1.92856424862784\\
0.0835	1.92844231131285\\
0.0836	1.92832028632171\\
0.0837	1.92819817365441\\
0.0838	1.92807597331095\\
0.0839	1.92795368529133\\
0.084	1.92783130959556\\
0.0841	1.92770884622362\\
0.0842	1.92758629517553\\
0.0843	1.92746365645128\\
0.0844	1.92734093005088\\
0.0845	1.92721811597431\\
0.0846	1.92709521422159\\
0.0847	1.92697222479271\\
0.0848	1.92684914768768\\
0.0849	1.92672598290648\\
0.085	1.92660273044913\\
0.0851	1.92647939031562\\
0.0852	1.92635596250595\\
0.0853	1.92623244702013\\
0.0854	1.92610884385815\\
0.0855	1.92598515302001\\
0.0856	1.92586137450571\\
0.0857	1.92573750831525\\
0.0858	1.92561355444864\\
0.0859	1.92548951290587\\
0.086	1.92536538368694\\
0.0861	1.92524116679185\\
0.0862	1.92511686222061\\
0.0863	1.92499246997321\\
0.0864	1.92486799004965\\
0.0865	1.92474342244993\\
0.0866	1.92461876717406\\
0.0867	1.92449402422202\\
0.0868	1.92436919359383\\
0.0869	1.92424427528949\\
0.087	1.92411926930898\\
0.0871	1.92399417565232\\
0.0872	1.9238689943195\\
0.0873	1.92374372531052\\
0.0874	1.92361836862538\\
0.0875	1.92349292426409\\
0.0876	1.92336739222664\\
0.0877	1.92324177251303\\
0.0878	1.92311606512326\\
0.0879	1.92299027005734\\
0.088	1.92286438731525\\
0.0881	1.92273841689701\\
0.0882	1.92261235880262\\
0.0883	1.92248621303206\\
0.0884	1.92235997958535\\
0.0885	1.92223365846248\\
0.0886	1.92210724966345\\
0.0887	1.92198075318826\\
0.0888	1.92185416903692\\
0.0889	1.92172749720942\\
0.089	1.92160073770576\\
0.0891	1.92147389052594\\
0.0892	1.92134695566997\\
0.0893	1.92121993313783\\
0.0894	1.92109282292955\\
0.0895	1.9209656250451\\
0.0896	1.92083833948449\\
0.0897	1.92071096624773\\
0.0898	1.92058350533481\\
0.0899	1.92045595674573\\
0.09	1.9203283204805\\
0.0901	1.9202005965391\\
0.0902	1.92007278492155\\
0.0903	1.91994488562784\\
0.0904	1.91981689865798\\
0.0905	1.91968882401195\\
0.0906	1.91956066168977\\
0.0907	1.91943241169143\\
0.0908	1.91930407401693\\
0.0909	1.91917564866628\\
0.091	1.91904713563946\\
0.0911	1.91891853493649\\
0.0912	1.91878984655737\\
0.0913	1.91866107050208\\
0.0914	1.91853220677064\\
0.0915	1.91840325536304\\
0.0916	1.91827421627928\\
0.0917	1.91814508951936\\
0.0918	1.91801587508329\\
0.0919	1.91788657297106\\
0.092	1.91775718318267\\
0.0921	1.91762770571812\\
0.0922	1.91749814057741\\
0.0923	1.91736848776055\\
0.0924	1.91723874726753\\
0.0925	1.91710891909835\\
0.0926	1.91697900325302\\
0.0927	1.91684899973153\\
0.0928	1.91671890853387\\
0.0929	1.91658872966007\\
0.093	1.9164584631101\\
0.0931	1.91632810888398\\
0.0932	1.91619766698169\\
0.0933	1.91606713740326\\
0.0934	1.91593652014866\\
0.0935	1.9158058152179\\
0.0936	1.91567502261099\\
0.0937	1.91554414232792\\
0.0938	1.91541317436869\\
0.0939	1.91528211873331\\
0.094	1.91515097542177\\
0.0941	1.91501974443407\\
0.0942	1.91488842577021\\
0.0943	1.91475701943019\\
0.0944	1.91462552541402\\
0.0945	1.91449394372169\\
0.0946	1.9143622743532\\
0.0947	1.91423051730855\\
0.0948	1.91409867258775\\
0.0949	1.91396674019078\\
0.095	1.91383472011766\\
0.0951	1.91370261236839\\
0.0952	1.91357041694295\\
0.0953	1.91343813384136\\
0.0954	1.91330576306361\\
0.0955	1.9131733046097\\
0.0956	1.91304075847963\\
0.0957	1.91290812467341\\
0.0958	1.91277540319103\\
0.0959	1.91264259403249\\
0.096	1.91250969719779\\
0.0961	1.91237671268694\\
0.0962	1.91224364049993\\
0.0963	1.91211048063676\\
0.0964	1.91197723309743\\
0.0965	1.91184389788195\\
0.0966	1.9117104749903\\
0.0967	1.9115769644225\\
0.0968	1.91144336617855\\
0.0969	1.91130968025843\\
0.097	1.91117590666216\\
0.0971	1.91104204538973\\
0.0972	1.91090809644114\\
0.0973	1.91077405981639\\
0.0974	1.91063993551549\\
0.0975	1.91050572353843\\
0.0976	1.91037142388521\\
0.0977	1.91023703655583\\
0.0978	1.91010256155029\\
0.0979	1.9099679988686\\
0.098	1.90983334851075\\
0.0981	1.90969861047674\\
0.0982	1.90956378476658\\
0.0983	1.90942887138026\\
0.0984	1.90929387031777\\
0.0985	1.90915878157914\\
0.0986	1.90902360516434\\
0.0987	1.90888834107339\\
0.0988	1.90875298930627\\
0.0989	1.90861754986301\\
0.099	1.90848202274358\\
0.0991	1.90834640794799\\
0.0992	1.90821070547625\\
0.0993	1.90807491532835\\
0.0994	1.90793903750429\\
0.0995	1.90780307200408\\
0.0996	1.90766701882771\\
0.0997	1.90753087797518\\
0.0998	1.90739464944649\\
0.0999	1.90725833324164\\
0.1	1.90712192936064\\
0.1001	1.90698543780348\\
0.1002	1.90684885857016\\
0.1003	1.90671219166068\\
0.1004	1.90657543707505\\
0.1005	1.90643859481325\\
0.1006	1.9063016648753\\
0.1007	1.9061646472612\\
0.1008	1.90602754197093\\
0.1009	1.90589034900451\\
0.101	1.90575306836193\\
0.1011	1.90561570004319\\
0.1012	1.9054782440483\\
0.1013	1.90534070037724\\
0.1014	1.90520306903003\\
0.1015	1.90506535000666\\
0.1016	1.90492754330714\\
0.1017	1.90478964893145\\
0.1018	1.90465166687961\\
0.1019	1.90451359715161\\
0.102	1.90437543974745\\
0.1021	1.90423719466714\\
0.1022	1.90409886191067\\
0.1023	1.90396044147804\\
0.1024	1.90382193336925\\
0.1025	1.9036833375843\\
0.1026	1.9035446541232\\
0.1027	1.90340588298594\\
0.1028	1.90326702417252\\
0.1029	1.90312807768294\\
0.103	1.90298904351721\\
0.1031	1.90284992167532\\
0.1032	1.90271071215727\\
0.1033	1.90257141496306\\
0.1034	1.9024320300927\\
0.1035	1.90229255754617\\
0.1036	1.90215299732349\\
0.1037	1.90201334942466\\
0.1038	1.90187361384966\\
0.1039	1.90173379059851\\
0.104	1.9015938796712\\
0.1041	1.90145388106773\\
0.1042	1.9013137947881\\
0.1043	1.90117362083232\\
0.1044	1.90103335920038\\
0.1045	1.90089300989228\\
0.1046	1.90075257290802\\
0.1047	1.90061204824761\\
0.1048	1.90047143591103\\
0.1049	1.9003307358983\\
0.105	1.90018994820942\\
0.1051	1.90004907284437\\
0.1052	1.89990810980317\\
0.1053	1.89976705908581\\
0.1054	1.89962592069229\\
0.1055	1.89948469462261\\
0.1056	1.89934338087678\\
0.1057	1.89920197945479\\
0.1058	1.89906049035664\\
0.1059	1.89891891358233\\
0.106	1.89877724913187\\
0.1061	1.89863549700525\\
0.1062	1.89849365720247\\
0.1063	1.89835172972353\\
0.1064	1.89820971456844\\
0.1065	1.89806761173718\\
0.1066	1.89792542122977\\
0.1067	1.8977831430462\\
0.1068	1.89764077718648\\
0.1069	1.8974983236506\\
0.107	1.89735578243855\\
0.1071	1.89721315355036\\
0.1072	1.897070436986\\
0.1073	1.89692763274548\\
0.1074	1.89678474082881\\
0.1075	1.89664176123598\\
0.1076	1.896498693967\\
0.1077	1.89635553902185\\
0.1078	1.89621229640055\\
0.1079	1.89606896610309\\
0.108	1.89592554812947\\
0.1081	1.8957820424797\\
0.1082	1.89563844915376\\
0.1083	1.89549476815167\\
0.1084	1.89535099947342\\
0.1085	1.89520714311902\\
0.1086	1.89506319908845\\
0.1087	1.89491916738173\\
0.1088	1.89477504799885\\
0.1089	1.89463084093982\\
0.109	1.89448654620462\\
0.1091	1.89434216379327\\
0.1092	1.89419769370576\\
0.1093	1.89405313594209\\
0.1094	1.89390849050227\\
0.1095	1.89376375738628\\
0.1096	1.89361893659414\\
0.1097	1.89347402812584\\
0.1098	1.89332903198139\\
0.1099	1.89318394816077\\
0.11	1.893038776664\\
0.1101	1.89289351749107\\
0.1102	1.89274817064199\\
0.1103	1.89260273611674\\
0.1104	1.89245721391534\\
0.1105	1.89231160403778\\
0.1106	1.89216590648406\\
0.1107	1.89202012125419\\
0.1108	1.89187424834815\\
0.1109	1.89172828776596\\
0.111	1.89158223950762\\
0.1111	1.89143610357311\\
0.1112	1.89128987996245\\
0.1113	1.89114356867562\\
0.1114	1.89099716971265\\
0.1115	1.89085068307351\\
0.1116	1.89070410875822\\
0.1117	1.89055744676676\\
0.1118	1.89041069709915\\
0.1119	1.89026385975539\\
0.112	1.89011693473546\\
0.1121	1.88996992203938\\
0.1122	1.88982282166714\\
0.1123	1.88967563361874\\
0.1124	1.88952835789418\\
0.1125	1.88938099449347\\
0.1126	1.8892335434166\\
0.1127	1.88908600466357\\
0.1128	1.88893837823438\\
0.1129	1.88879066412904\\
0.113	1.88864286234754\\
0.1131	1.88849497288988\\
0.1132	1.88834699575606\\
0.1133	1.88819893094609\\
0.1134	1.88805077845996\\
0.1135	1.88790253829767\\
0.1136	1.88775421045922\\
0.1137	1.88760579494461\\
0.1138	1.88745729175385\\
0.1139	1.88730870088693\\
0.114	1.88716002234385\\
0.1141	1.88701125612461\\
0.1142	1.88686240222922\\
0.1143	1.88671346065767\\
0.1144	1.88656443140996\\
0.1145	1.88641531448609\\
0.1146	1.88626610988607\\
0.1147	1.88611681760988\\
0.1148	1.88596743765754\\
0.1149	1.88581797002905\\
0.115	1.88566841472439\\
0.1151	1.88551877174358\\
0.1152	1.88536904108661\\
0.1153	1.88521922275348\\
0.1154	1.88506931674419\\
0.1155	1.88491932305875\\
0.1156	1.88476924169715\\
0.1157	1.88461907265939\\
0.1158	1.88446881594547\\
0.1159	1.8843184715554\\
0.116	1.88416803948917\\
0.1161	1.88401751974678\\
0.1162	1.88386691232823\\
0.1163	1.88371621723352\\
0.1164	1.88356543446266\\
0.1165	1.88341456401564\\
0.1166	1.88326360589246\\
0.1167	1.88311256009313\\
0.1168	1.88296142661763\\
0.1169	1.88281020546598\\
0.117	1.88265889663817\\
0.1171	1.88250750013421\\
0.1172	1.88235601595408\\
0.1173	1.8822044440978\\
0.1174	1.88205278456536\\
0.1175	1.88190103735676\\
0.1176	1.88174920247201\\
0.1177	1.8815972799111\\
0.1178	1.88144526967403\\
0.1179	1.8812931717608\\
0.118	1.88114098617141\\
0.1181	1.88098871290587\\
0.1182	1.88083635196417\\
0.1183	1.88068390334631\\
0.1184	1.88053136705229\\
0.1185	1.88037874308212\\
0.1186	1.88022603143579\\
0.1187	1.8800732321133\\
0.1188	1.87992034511465\\
0.1189	1.87976737043985\\
0.119	1.87961430808888\\
0.1191	1.87946115806176\\
0.1192	1.87930792035849\\
0.1193	1.87915459497905\\
0.1194	1.87900118192346\\
0.1195	1.87884768119171\\
0.1196	1.8786940927838\\
0.1197	1.87854041669973\\
0.1198	1.87838665293951\\
0.1199	1.87823280150313\\
0.12	1.87807886239059\\
0.1201	1.87792483560189\\
0.1202	1.87777072113704\\
0.1203	1.87761651899602\\
0.1204	1.87746222917885\\
0.1205	1.87730785168553\\
0.1206	1.87715338651604\\
0.1207	1.8769988336704\\
0.1208	1.8768441931486\\
0.1209	1.87668946495064\\
0.121	1.87653464907652\\
0.1211	1.87637974552625\\
0.1212	1.87622475429982\\
0.1213	1.87606967539723\\
0.1214	1.87591450881848\\
0.1215	1.87575925456358\\
0.1216	1.87560391263252\\
0.1217	1.8754484830253\\
0.1218	1.87529296574192\\
0.1219	1.87513736078238\\
0.122	1.87498166814669\\
0.1221	1.87482588783484\\
0.1222	1.87467001984683\\
0.1223	1.87451406418267\\
0.1224	1.87435802084234\\
0.1225	1.87420188982586\\
0.1226	1.87404567113322\\
0.1227	1.87388936476443\\
0.1228	1.87373297071947\\
0.1229	1.87357648899836\\
0.123	1.87341991960109\\
0.1231	1.87326326252766\\
0.1232	1.87310651777808\\
0.1233	1.87294968535234\\
0.1234	1.87279276525044\\
0.1235	1.87263575747238\\
0.1236	1.87247866201816\\
0.1237	1.87232147888779\\
0.1238	1.87216420808126\\
0.1239	1.87200684959857\\
0.124	1.87184940343972\\
0.1241	1.87169186960472\\
0.1242	1.87153424809356\\
0.1243	1.87137653890624\\
0.1244	1.87121874204276\\
0.1245	1.87106085750313\\
0.1246	1.87090288528733\\
0.1247	1.87074482539538\\
0.1248	1.87058667782728\\
0.1249	1.87042844258301\\
0.125	1.87027011966259\\
0.1251	1.87011170906601\\
0.1252	1.86995321079327\\
0.1253	1.86979462484437\\
0.1254	1.86963595121932\\
0.1255	1.86947718991811\\
0.1256	1.86931834094074\\
0.1257	1.86915940428721\\
0.1258	1.86900037995753\\
0.1259	1.86884126795168\\
0.126	1.86868206826968\\
0.1261	1.86852278091153\\
0.1262	1.86836340587721\\
0.1263	1.86820394316674\\
0.1264	1.86804439278011\\
0.1265	1.86788475471732\\
0.1266	1.86772502897837\\
0.1267	1.86756521556327\\
0.1268	1.86740531447201\\
0.1269	1.86724532570459\\
0.127	1.86708524926101\\
0.1271	1.86692508514128\\
0.1272	1.86676483334539\\
0.1273	1.86660449387334\\
0.1274	1.86644406672513\\
0.1275	1.86628355190076\\
0.1276	1.86612294940024\\
0.1277	1.86596225922356\\
0.1278	1.86580148137072\\
0.1279	1.86564061584173\\
0.128	1.86547966263657\\
0.1281	1.86531862175526\\
0.1282	1.86515749319779\\
0.1283	1.86499627696417\\
0.1284	1.86483497305438\\
0.1285	1.86467358146844\\
0.1286	1.86451210220634\\
0.1287	1.86435053526808\\
0.1288	1.86418888065367\\
0.1289	1.8640271383631\\
0.129	1.86386530839637\\
0.1291	1.86370339075348\\
0.1292	1.86354138543443\\
0.1293	1.86337929243923\\
0.1294	1.86321711176787\\
0.1295	1.86305484342035\\
0.1296	1.86289248739667\\
0.1297	1.86273004369684\\
0.1298	1.86256751232085\\
0.1299	1.8624048932687\\
0.13	1.86224218654039\\
0.1301	1.86207939213593\\
0.1302	1.86191651005531\\
0.1303	1.86175354029853\\
0.1304	1.86159048286559\\
0.1305	1.86142733775649\\
0.1306	1.86126410497124\\
0.1307	1.86110078450983\\
0.1308	1.86093737637226\\
0.1309	1.86077388055853\\
0.131	1.86061029706865\\
0.1311	1.86044662590261\\
0.1312	1.86028286706041\\
0.1313	1.86011902054205\\
0.1314	1.85995508634754\\
0.1315	1.85979106447687\\
0.1316	1.85962695493004\\
0.1317	1.85946275770705\\
0.1318	1.8592984728079\\
0.1319	1.8591341002326\\
0.132	1.85896963998114\\
0.1321	1.85880509205352\\
0.1322	1.85864045644975\\
0.1323	1.85847573316981\\
0.1324	1.85831092221372\\
0.1325	1.85814602358147\\
0.1326	1.85798103727307\\
0.1327	1.8578159632885\\
0.1328	1.85765080162778\\
0.1329	1.8574855522909\\
0.133	1.85732021527786\\
0.1331	1.85715479058867\\
0.1332	1.85698927822332\\
0.1333	1.85682367818181\\
0.1334	1.85665799046414\\
0.1335	1.85649221507031\\
0.1336	1.85632635200033\\
0.1337	1.85616040125419\\
0.1338	1.85599436283189\\
0.1339	1.85582823673343\\
0.134	1.85566202295882\\
0.1341	1.85549572150805\\
0.1342	1.85532933238112\\
0.1343	1.85516285557803\\
0.1344	1.85499629109878\\
0.1345	1.85482963894338\\
0.1346	1.85466289911182\\
0.1347	1.8544960716041\\
0.1348	1.85432915642023\\
0.1349	1.8541621535602\\
0.135	1.853995063024\\
0.1351	1.85382788481166\\
0.1352	1.85366061892315\\
0.1353	1.85349326535849\\
0.1354	1.85332582411766\\
0.1355	1.85315829520068\\
0.1356	1.85299067860755\\
0.1357	1.85282297433825\\
0.1358	1.8526551823928\\
0.1359	1.85248730277119\\
0.136	1.85231933547342\\
0.1361	1.8521512804995\\
0.1362	1.85198313784941\\
0.1363	1.85181490752317\\
0.1364	1.85164658952078\\
0.1365	1.85147818384222\\
0.1366	1.85130969048751\\
0.1367	1.85114110945663\\
0.1368	1.85097244074961\\
0.1369	1.85080368436642\\
0.137	1.85063484030707\\
0.1371	1.85046590857157\\
0.1372	1.85029688915991\\
0.1373	1.85012778207209\\
0.1374	1.84995858730812\\
0.1375	1.84978930486799\\
0.1376	1.8496199347517\\
0.1377	1.84945047695925\\
0.1378	1.84928093149064\\
0.1379	1.84911129834588\\
0.138	1.84894157752496\\
0.1381	1.84877176902788\\
0.1382	1.84860187285464\\
0.1383	1.84843188900525\\
0.1384	1.8482618174797\\
0.1385	1.84809165827799\\
0.1386	1.84792141140012\\
0.1387	1.84775107684609\\
0.1388	1.84758065461591\\
0.1389	1.84741014470957\\
0.139	1.84723954712707\\
0.1391	1.84706886186842\\
0.1392	1.8468980889336\\
0.1393	1.84672722832263\\
0.1394	1.8465562800355\\
0.1395	1.84638524407222\\
0.1396	1.84621412043277\\
0.1397	1.84604290911717\\
0.1398	1.84587161012541\\
0.1399	1.8457002234575\\
0.14	1.84552874911342\\
};
\addlegendentry{Best-Fit Solution};

\addplot [color=mycolor3,solid]
  table[row sep=crcr]{%
0	2.00805848758789\\
0.0001	2.00797873611462\\
0.0002	2.00789894514725\\
0.0003	2.00781911468578\\
0.0004	2.00773924473021\\
0.0005	2.00765933528053\\
0.0006	2.00757938633675\\
0.0007	2.00749939789887\\
0.0008	2.00741936996688\\
0.0009	2.0073393025408\\
0.001	2.00725919562061\\
0.0011	2.00717904920631\\
0.0012	2.00709886329792\\
0.0013	2.00701863789542\\
0.0014	2.00693837299882\\
0.0015	2.00685806860812\\
0.0016	2.00677772472332\\
0.0017	2.00669734134441\\
0.0018	2.0066169184714\\
0.0019	2.00653645610429\\
0.002	2.00645595424308\\
0.0021	2.00637541288776\\
0.0022	2.00629483203834\\
0.0023	2.00621421169482\\
0.0024	2.0061335518572\\
0.0025	2.00605285252547\\
0.0026	2.00597211369964\\
0.0027	2.00589133537971\\
0.0028	2.00581051756568\\
0.0029	2.00572966025754\\
0.003	2.0056487634553\\
0.0031	2.00556782715896\\
0.0032	2.00548685136852\\
0.0033	2.00540583608397\\
0.0034	2.00532478130532\\
0.0035	2.00524368703257\\
0.0036	2.00516255326572\\
0.0037	2.00508138000477\\
0.0038	2.00500016724971\\
0.0039	2.00491891500055\\
0.004	2.00483762325728\\
0.0041	2.00475629201992\\
0.0042	2.00467492128845\\
0.0043	2.00459351106288\\
0.0044	2.00451206134321\\
0.0045	2.00443057212943\\
0.0046	2.00434904342155\\
0.0047	2.00426747521957\\
0.0048	2.00418586752349\\
0.0049	2.0041042203333\\
0.005	2.00402253364902\\
0.0051	2.00394080747063\\
0.0052	2.00385904179813\\
0.0053	2.00377723663154\\
0.0054	2.00369539197084\\
0.0055	2.00361350781604\\
0.0056	2.00353158416714\\
0.0057	2.00344962102414\\
0.0058	2.00336761838703\\
0.0059	2.00328557625582\\
0.006	2.00320349463051\\
0.0061	2.00312137351109\\
0.0062	2.00303921289758\\
0.0063	2.00295701278996\\
0.0064	2.00287477318823\\
0.0065	2.00279249409241\\
0.0066	2.00271017550248\\
0.0067	2.00262781741845\\
0.0068	2.00254541984032\\
0.0069	2.00246298276809\\
0.007	2.00238050620175\\
0.0071	2.00229799014131\\
0.0072	2.00221543458677\\
0.0073	2.00213283953813\\
0.0074	2.00205020499538\\
0.0075	2.00196753095853\\
0.0076	2.00188481742758\\
0.0077	2.00180206440253\\
0.0078	2.00171927188337\\
0.0079	2.00163643987011\\
0.008	2.00155356836275\\
0.0081	2.00147065736129\\
0.0082	2.00138770686572\\
0.0083	2.00130471687605\\
0.0084	2.00122168739228\\
0.0085	2.00113861841441\\
0.0086	2.00105550994243\\
0.0087	2.00097236197635\\
0.0088	2.00088917451617\\
0.0089	2.00080594756189\\
0.009	2.0007226811135\\
0.0091	2.00063937517102\\
0.0092	2.00055602973442\\
0.0093	2.00047264480373\\
0.0094	2.00038922037894\\
0.0095	2.00030575646004\\
0.0096	2.00022225304704\\
0.0097	2.00013871013994\\
0.0098	2.00005512773873\\
0.0099	1.99997150584342\\
0.01	1.99988784445401\\
0.0101	1.9998041435705\\
0.0102	1.99972040319288\\
0.0103	1.99963662332116\\
0.0104	1.99955280395535\\
0.0105	1.99946894509542\\
0.0106	1.9993850467414\\
0.0107	1.99930110889327\\
0.0108	1.99921713155104\\
0.0109	1.99913311471471\\
0.011	1.99904905838427\\
0.0111	1.99896496255974\\
0.0112	1.9988808272411\\
0.0113	1.99879665242835\\
0.0114	1.99871243812151\\
0.0115	1.99862818432056\\
0.0116	1.99854389102551\\
0.0117	1.99845955823636\\
0.0118	1.99837518595311\\
0.0119	1.99829077417575\\
0.012	1.99820632290429\\
0.0121	1.99812183213873\\
0.0122	1.99803730187906\\
0.0123	1.9979527321253\\
0.0124	1.99786812287743\\
0.0125	1.99778347413546\\
0.0126	1.99769878589938\\
0.0127	1.99761405816921\\
0.0128	1.99752929094493\\
0.0129	1.99744448422655\\
0.013	1.99735963801406\\
0.0131	1.99727475230748\\
0.0132	1.99718982710679\\
0.0133	1.997104862412\\
0.0134	1.9970198582231\\
0.0135	1.99693481454011\\
0.0136	1.99684973136301\\
0.0137	1.99676460869181\\
0.0138	1.9966794465265\\
0.0139	1.9965942448671\\
0.014	1.99650900371359\\
0.0141	1.99642372306598\\
0.0142	1.99633840292426\\
0.0143	1.99625304328845\\
0.0144	1.99616764415853\\
0.0145	1.99608220553451\\
0.0146	1.99599672741639\\
0.0147	1.99591120980416\\
0.0148	1.99582565269783\\
0.0149	1.9957400560974\\
0.015	1.99565442000287\\
0.0151	1.99556874441423\\
0.0152	1.9954830293315\\
0.0153	1.99539727475466\\
0.0154	1.99531148068371\\
0.0155	1.99522564711867\\
0.0156	1.99513977405952\\
0.0157	1.99505386150627\\
0.0158	1.99496790945892\\
0.0159	1.99488191791746\\
0.016	1.99479588688191\\
0.0161	1.99470981635224\\
0.0162	1.99462370632848\\
0.0163	1.99453755681062\\
0.0164	1.99445136779865\\
0.0165	1.99436513929258\\
0.0166	1.99427887129241\\
0.0167	1.99419256379813\\
0.0168	1.99410621680976\\
0.0169	1.99401983032728\\
0.017	1.9939334043507\\
0.0171	1.99384693888001\\
0.0172	1.99376043391522\\
0.0173	1.99367388945634\\
0.0174	1.99358730550334\\
0.0175	1.99350068205625\\
0.0176	1.99341401911505\\
0.0177	1.99332731667975\\
0.0178	1.99324057475035\\
0.0179	1.99315379332685\\
0.018	1.99306697240924\\
0.0181	1.99298011199753\\
0.0182	1.99289321209172\\
0.0183	1.99280627269181\\
0.0184	1.99271929379779\\
0.0185	1.99263227540967\\
0.0186	1.99254521752745\\
0.0187	1.99245812015113\\
0.0188	1.9923709832807\\
0.0189	1.99228380691617\\
0.019	1.99219659105754\\
0.0191	1.99210933570481\\
0.0192	1.99202204085797\\
0.0193	1.99193470651703\\
0.0194	1.99184733268199\\
0.0195	1.99175991935285\\
0.0196	1.9916724665296\\
0.0197	1.99158497421225\\
0.0198	1.9914974424008\\
0.0199	1.99140987109525\\
0.02	1.99132226029559\\
0.0201	1.99123461000184\\
0.0202	1.99114692021398\\
0.0203	1.99105919093201\\
0.0204	1.99097142215595\\
0.0205	1.99088361388578\\
0.0206	1.99079576612151\\
0.0207	1.99070787886314\\
0.0208	1.99061995211066\\
0.0209	1.99053198586408\\
0.021	1.9904439801234\\
0.0211	1.99035593488862\\
0.0212	1.99026785015974\\
0.0213	1.99017972593675\\
0.0214	1.99009156221966\\
0.0215	1.99000335900847\\
0.0216	1.98991511630317\\
0.0217	1.98982683410377\\
0.0218	1.98973851241027\\
0.0219	1.98965015122267\\
0.022	1.98956175054097\\
0.0221	1.98947331036516\\
0.0222	1.98938483069525\\
0.0223	1.98929631153124\\
0.0224	1.98920775287312\\
0.0225	1.98911915472091\\
0.0226	1.98903051707459\\
0.0227	1.98894183993417\\
0.0228	1.98885312329964\\
0.0229	1.98876436717101\\
0.023	1.98867557154829\\
0.0231	1.98858673643145\\
0.0232	1.98849786182052\\
0.0233	1.98840894771548\\
0.0234	1.98831999411634\\
0.0235	1.9882310010231\\
0.0236	1.98814196843576\\
0.0237	1.98805289635431\\
0.0238	1.98796378477876\\
0.0239	1.98787463370911\\
0.024	1.98778544314536\\
0.0241	1.9876962130875\\
0.0242	1.98760694353554\\
0.0243	1.98751763448948\\
0.0244	1.98742828594932\\
0.0245	1.98733889791505\\
0.0246	1.98724947038668\\
0.0247	1.98716000336421\\
0.0248	1.98707049684764\\
0.0249	1.98698095083696\\
0.025	1.98689136533219\\
0.0251	1.9868017403333\\
0.0252	1.98671207584032\\
0.0253	1.98662237185324\\
0.0254	1.98653262837205\\
0.0255	1.98644284539676\\
0.0256	1.98635302292736\\
0.0257	1.98626316096387\\
0.0258	1.98617325950627\\
0.0259	1.98608331855457\\
0.026	1.98599333810877\\
0.0261	1.98590331816886\\
0.0262	1.98581325873485\\
0.0263	1.98572315980674\\
0.0264	1.98563302138453\\
0.0265	1.98554284346822\\
0.0266	1.9854526260578\\
0.0267	1.98536236915328\\
0.0268	1.98527207275466\\
0.0269	1.98518173686193\\
0.027	1.9850913614751\\
0.0271	1.98500094659418\\
0.0272	1.98491049221914\\
0.0273	1.98481999835001\\
0.0274	1.98472946498677\\
0.0275	1.98463889212943\\
0.0276	1.98454827977799\\
0.0277	1.98445762793244\\
0.0278	1.9843669365928\\
0.0279	1.98427620575905\\
0.028	1.9841854354312\\
0.0281	1.98409462560924\\
0.0282	1.98400377629318\\
0.0283	1.98391288748302\\
0.0284	1.98382195917876\\
0.0285	1.9837309913804\\
0.0286	1.98363998408793\\
0.0287	1.98354893730136\\
0.0288	1.98345785102069\\
0.0289	1.98336672524592\\
0.029	1.98327555997704\\
0.0291	1.98318435521406\\
0.0292	1.98309311095698\\
0.0293	1.9830018272058\\
0.0294	1.98291050396051\\
0.0295	1.98281914122112\\
0.0296	1.98272773898763\\
0.0297	1.98263629726004\\
0.0298	1.98254481603834\\
0.0299	1.98245329532254\\
0.03	1.98236173511264\\
0.0301	1.98227013540864\\
0.0302	1.98217849621053\\
0.0303	1.98208681751832\\
0.0304	1.98199509933201\\
0.0305	1.9819033416516\\
0.0306	1.98181154447708\\
0.0307	1.98171970780847\\
0.0308	1.98162783164575\\
0.0309	1.98153591598892\\
0.031	1.981443960838\\
0.0311	1.98135196619297\\
0.0312	1.98125993205384\\
0.0313	1.98116785842061\\
0.0314	1.98107574529327\\
0.0315	1.98098359267183\\
0.0316	1.98089140055629\\
0.0317	1.98079916894665\\
0.0318	1.9807068978429\\
0.0319	1.98061458724506\\
0.032	1.98052223715311\\
0.0321	1.98042984756705\\
0.0322	1.9803374184869\\
0.0323	1.98024494991264\\
0.0324	1.98015244184428\\
0.0325	1.98005989428182\\
0.0326	1.97996730722525\\
0.0327	1.97987468067459\\
0.0328	1.97978201462982\\
0.0329	1.97968930909095\\
0.033	1.97959656405797\\
0.0331	1.97950377953089\\
0.0332	1.97941095550971\\
0.0333	1.97931809199443\\
0.0334	1.97922518898505\\
0.0335	1.97913224648156\\
0.0336	1.97903926448397\\
0.0337	1.97894624299228\\
0.0338	1.97885318200649\\
0.0339	1.97876008152659\\
0.034	1.97866694155259\\
0.0341	1.97857376208449\\
0.0342	1.97848054312228\\
0.0343	1.97838728466598\\
0.0344	1.97829398671557\\
0.0345	1.97820064927106\\
0.0346	1.97810727233244\\
0.0347	1.97801385589973\\
0.0348	1.97792039997291\\
0.0349	1.97782690455199\\
0.035	1.97773336963696\\
0.0351	1.97763979522784\\
0.0352	1.97754618132461\\
0.0353	1.97745252792728\\
0.0354	1.97735883503584\\
0.0355	1.97726510265031\\
0.0356	1.97717133077067\\
0.0357	1.97707751939693\\
0.0358	1.97698366852909\\
0.0359	1.97688977816714\\
0.036	1.97679584831109\\
0.0361	1.97670187896094\\
0.0362	1.97660787011669\\
0.0363	1.97651382177833\\
0.0364	1.97641973394588\\
0.0365	1.97632560661931\\
0.0366	1.97623143979865\\
0.0367	1.97613723348389\\
0.0368	1.97604298767502\\
0.0369	1.97594870237205\\
0.037	1.97585437757498\\
0.0371	1.9757600132838\\
0.0372	1.97566560949852\\
0.0373	1.97557116621914\\
0.0374	1.97547668344566\\
0.0375	1.97538216117807\\
0.0376	1.97528759941639\\
0.0377	1.9751929981606\\
0.0378	1.9750983574107\\
0.0379	1.97500367716671\\
0.038	1.97490895742861\\
0.0381	1.97481419819641\\
0.0382	1.97471939947011\\
0.0383	1.97462456124971\\
0.0384	1.9745296835352\\
0.0385	1.97443476632659\\
0.0386	1.97433980962388\\
0.0387	1.97424481342706\\
0.0388	1.97414977773615\\
0.0389	1.97405470255113\\
0.039	1.973959587872\\
0.0391	1.97386443369878\\
0.0392	1.97376924003145\\
0.0393	1.97367400687002\\
0.0394	1.97357873421449\\
0.0395	1.97348342206486\\
0.0396	1.97338807042112\\
0.0397	1.97329267928328\\
0.0398	1.97319724865134\\
0.0399	1.9731017785253\\
0.04	1.97300626890515\\
0.0401	1.9729107197909\\
0.0402	1.97281513118255\\
0.0403	1.9727195030801\\
0.0404	1.97262383548354\\
0.0405	1.97252812839288\\
0.0406	1.97243238180812\\
0.0407	1.97233659572926\\
0.0408	1.97224077015629\\
0.0409	1.97214490508922\\
0.041	1.97204900052805\\
0.0411	1.97195305647278\\
0.0412	1.9718570729234\\
0.0413	1.97176104987993\\
0.0414	1.97166498734234\\
0.0415	1.97156888531066\\
0.0416	1.97147274378488\\
0.0417	1.97137656276499\\
0.0418	1.971280342251\\
0.0419	1.9711840822429\\
0.042	1.97108778274071\\
0.0421	1.97099144374441\\
0.0422	1.97089506525401\\
0.0423	1.97079864726951\\
0.0424	1.9707021897909\\
0.0425	1.9706056928182\\
0.0426	1.97050915635138\\
0.0427	1.97041258039047\\
0.0428	1.97031596493546\\
0.0429	1.97021930998634\\
0.043	1.97012261554312\\
0.0431	1.9700258816058\\
0.0432	1.96992910817437\\
0.0433	1.96983229524884\\
0.0434	1.96973544282922\\
0.0435	1.96963855091548\\
0.0436	1.96954161950765\\
0.0437	1.96944464860571\\
0.0438	1.96934763820967\\
0.0439	1.96925058831953\\
0.044	1.96915349893529\\
0.0441	1.96905637005694\\
0.0442	1.96895920168449\\
0.0443	1.96886199381794\\
0.0444	1.96876474645728\\
0.0445	1.96866745960253\\
0.0446	1.96857013325367\\
0.0447	1.96847276741071\\
0.0448	1.96837536207364\\
0.0449	1.96827791724247\\
0.045	1.96818043291721\\
0.0451	1.96808290909783\\
0.0452	1.96798534578436\\
0.0453	1.96788774297678\\
0.0454	1.9677901006751\\
0.0455	1.96769241887932\\
0.0456	1.96759469758944\\
0.0457	1.96749693680545\\
0.0458	1.96739913652736\\
0.0459	1.96730129675517\\
0.046	1.96720341748888\\
0.0461	1.96710549872848\\
0.0462	1.96700754047399\\
0.0463	1.96690954272538\\
0.0464	1.96681150548268\\
0.0465	1.96671342874588\\
0.0466	1.96661531251497\\
0.0467	1.96651715678996\\
0.0468	1.96641896157084\\
0.0469	1.96632072685763\\
0.047	1.96622245265031\\
0.0471	1.96612413894889\\
0.0472	1.96602578575337\\
0.0473	1.96592739306374\\
0.0474	1.96582896088001\\
0.0475	1.96573048920218\\
0.0476	1.96563197803025\\
0.0477	1.96553342736421\\
0.0478	1.96543483720408\\
0.0479	1.96533620754984\\
0.048	1.96523753840149\\
0.0481	1.96513882975905\\
0.0482	1.9650400816225\\
0.0483	1.96494129399185\\
0.0484	1.9648424668671\\
0.0485	1.96474360024824\\
0.0486	1.96464469413529\\
0.0487	1.96454574852823\\
0.0488	1.96444676342706\\
0.0489	1.9643477388318\\
0.049	1.96424867474243\\
0.0491	1.96414957115896\\
0.0492	1.96405042808139\\
0.0493	1.96395124550971\\
0.0494	1.96385202344394\\
0.0495	1.96375276188406\\
0.0496	1.96365346083008\\
0.0497	1.96355412028199\\
0.0498	1.96345474023981\\
0.0499	1.96335532070352\\
0.05	1.96325586167312\\
0.0501	1.96315636314863\\
0.0502	1.96305682513003\\
0.0503	1.96295724761733\\
0.0504	1.96285763061053\\
0.0505	1.96275797410963\\
0.0506	1.96265827811462\\
0.0507	1.96255854262551\\
0.0508	1.9624587676423\\
0.0509	1.96235895316499\\
0.051	1.96225909919357\\
0.0511	1.96215920572805\\
0.0512	1.96205927276843\\
0.0513	1.96195930031471\\
0.0514	1.96185928836688\\
0.0515	1.96175923692495\\
0.0516	1.96165914598892\\
0.0517	1.96155901555879\\
0.0518	1.96145884563455\\
0.0519	1.96135863621622\\
0.052	1.96125838730377\\
0.0521	1.96115809889723\\
0.0522	1.96105777099658\\
0.0523	1.96095740360184\\
0.0524	1.96085699671299\\
0.0525	1.96075655033003\\
0.0526	1.96065606445298\\
0.0527	1.96055553908182\\
0.0528	1.96045497421656\\
0.0529	1.9603543698572\\
0.053	1.96025372600373\\
0.0531	1.96015304265616\\
0.0532	1.96005231981449\\
0.0533	1.95995155747872\\
0.0534	1.95985075564885\\
0.0535	1.95974991432487\\
0.0536	1.95964903350679\\
0.0537	1.95954811319461\\
0.0538	1.95944715338832\\
0.0539	1.95934615408793\\
0.054	1.95924511529344\\
0.0541	1.95914403700485\\
0.0542	1.95904291922216\\
0.0543	1.95894176194536\\
0.0544	1.95884056517446\\
0.0545	1.95873932890946\\
0.0546	1.95863805315035\\
0.0547	1.95853673789715\\
0.0548	1.95843538314984\\
0.0549	1.95833398890842\\
0.055	1.95823255517291\\
0.0551	1.95813108194329\\
0.0552	1.95802956921957\\
0.0553	1.95792801700175\\
0.0554	1.95782642528983\\
0.0555	1.9577247940838\\
0.0556	1.95762312338367\\
0.0557	1.95752141318944\\
0.0558	1.95741966350111\\
0.0559	1.95731787431867\\
0.056	1.95721604564213\\
0.0561	1.95711417747149\\
0.0562	1.95701226980674\\
0.0563	1.9569103226479\\
0.0564	1.95680833599495\\
0.0565	1.9567063098479\\
0.0566	1.95660424420674\\
0.0567	1.95650213907149\\
0.0568	1.95639999444213\\
0.0569	1.95629781031867\\
0.057	1.95619558670111\\
0.0571	1.95609332358944\\
0.0572	1.95599102098367\\
0.0573	1.9558886788838\\
0.0574	1.95578629728983\\
0.0575	1.95568387620175\\
0.0576	1.95558141561957\\
0.0577	1.95547891554329\\
0.0578	1.95537637597291\\
0.0579	1.95527379690842\\
0.058	1.95517117834984\\
0.0581	1.95506852029714\\
0.0582	1.95496582275035\\
0.0583	1.95486308570946\\
0.0584	1.95476030917446\\
0.0585	1.95465749314536\\
0.0586	1.95455463762216\\
0.0587	1.95445174260485\\
0.0588	1.95434880809344\\
0.0589	1.95424583408793\\
0.059	1.95414282058832\\
0.0591	1.95403976759461\\
0.0592	1.95393667510679\\
0.0593	1.95383354312487\\
0.0594	1.95373037164885\\
0.0595	1.95362716067872\\
0.0596	1.95352391021449\\
0.0597	1.95342062025616\\
0.0598	1.95331729080373\\
0.0599	1.9532139218572\\
0.06	1.95311051341656\\
0.0601	1.95300706548182\\
0.0602	1.95290357805298\\
0.0603	1.95280005113003\\
0.0604	1.95269648471299\\
0.0605	1.95259287880184\\
0.0606	1.95248923339659\\
0.0607	1.95238554849723\\
0.0608	1.95228182410377\\
0.0609	1.95217806021621\\
0.061	1.95207425683455\\
0.0611	1.95197041395879\\
0.0612	1.95186653158892\\
0.0613	1.95176260972495\\
0.0614	1.95165864836688\\
0.0615	1.95155464751471\\
0.0616	1.95145060716843\\
0.0617	1.95134652732805\\
0.0618	1.95124240799357\\
0.0619	1.95113824916499\\
0.062	1.9510340508423\\
0.0621	1.95092981302551\\
0.0622	1.95082553571462\\
0.0623	1.95072121890963\\
0.0624	1.95061686261053\\
0.0625	1.95051246681733\\
0.0626	1.95040803153003\\
0.0627	1.95030355674863\\
0.0628	1.95019904247312\\
0.0629	1.95009448870352\\
0.063	1.94998989543981\\
0.0631	1.94988526268199\\
0.0632	1.94978059043008\\
0.0633	1.94967587868406\\
0.0634	1.94957112744394\\
0.0635	1.94946633670971\\
0.0636	1.94936150648139\\
0.0637	1.94925663675896\\
0.0638	1.94915172754243\\
0.0639	1.9490467788318\\
0.064	1.94894179062706\\
0.0641	1.94883676292823\\
0.0642	1.94873169573528\\
0.0643	1.94862658904824\\
0.0644	1.9485214428671\\
0.0645	1.94841625719185\\
0.0646	1.9483110320225\\
0.0647	1.94820576735905\\
0.0648	1.94810046320149\\
0.0649	1.94799511954984\\
0.065	1.94788973640408\\
0.0651	1.94778431376421\\
0.0652	1.94767885163025\\
0.0653	1.94757335000218\\
0.0654	1.94746780888001\\
0.0655	1.94736222826374\\
0.0656	1.94725660815337\\
0.0657	1.94715094854889\\
0.0658	1.94704524945031\\
0.0659	1.94693951085763\\
0.066	1.94683373277084\\
0.0661	1.94672791518996\\
0.0662	1.94662205811497\\
0.0663	1.94651616154587\\
0.0664	1.94641022548268\\
0.0665	1.94630424992538\\
0.0666	1.94619823487399\\
0.0667	1.94609218032848\\
0.0668	1.94598608628888\\
0.0669	1.94587995275517\\
0.067	1.94577377972736\\
0.0671	1.94566756720545\\
0.0672	1.94556131518944\\
0.0673	1.94545502367932\\
0.0674	1.9453486926751\\
0.0675	1.94524232217678\\
0.0676	1.94513591218436\\
0.0677	1.94502946269783\\
0.0678	1.94492297371721\\
0.0679	1.94481644524247\\
0.068	1.94470987727364\\
0.0681	1.94460326981071\\
0.0682	1.94449662285367\\
0.0683	1.94438993640253\\
0.0684	1.94428321045728\\
0.0685	1.94417644501794\\
0.0686	1.94406964008449\\
0.0687	1.94396279565694\\
0.0688	1.94385591173529\\
0.0689	1.94374898831953\\
0.069	1.94364202540967\\
0.0691	1.94353502300571\\
0.0692	1.94342798110765\\
0.0693	1.94332089971548\\
0.0694	1.94321377882922\\
0.0695	1.94310661844885\\
0.0696	1.94299941857437\\
0.0697	1.9428921792058\\
0.0698	1.94278490034312\\
0.0699	1.94267758198634\\
0.07	1.94257022413546\\
0.0701	1.94246282679047\\
0.0702	1.94235538995139\\
0.0703	1.9422479136182\\
0.0704	1.9421403977909\\
0.0705	1.94203284246951\\
0.0706	1.94192524765401\\
0.0707	1.94181761334441\\
0.0708	1.94170993954071\\
0.0709	1.9416022262429\\
0.071	1.941494473451\\
0.0711	1.94138668116499\\
0.0712	1.94127884938488\\
0.0713	1.94117097811066\\
0.0714	1.94106306734235\\
0.0715	1.94095511707993\\
0.0716	1.9408471273234\\
0.0717	1.94073909807278\\
0.0718	1.94063102932805\\
0.0719	1.94052292108922\\
0.072	1.94041477335629\\
0.0721	1.94030658612926\\
0.0722	1.94019835940812\\
0.0723	1.94009009319288\\
0.0724	1.93998178748354\\
0.0725	1.9398734422801\\
0.0726	1.93976505758255\\
0.0727	1.9396566333909\\
0.0728	1.93954816970515\\
0.0729	1.9394396665253\\
0.073	1.93933112385134\\
0.0731	1.93922254168328\\
0.0732	1.93911392002112\\
0.0733	1.93900525886486\\
0.0734	1.93889655821449\\
0.0735	1.93878781807002\\
0.0736	1.93867903843145\\
0.0737	1.93857021929878\\
0.0738	1.938461360672\\
0.0739	1.93835246255113\\
0.074	1.93824352493615\\
0.0741	1.93813454782706\\
0.0742	1.93802553122388\\
0.0743	1.93791647512659\\
0.0744	1.9378073795352\\
0.0745	1.93769824444971\\
0.0746	1.93758906987011\\
0.0747	1.93747985579641\\
0.0748	1.93737060222861\\
0.0749	1.93726130916671\\
0.075	1.9371519766107\\
0.0751	1.9370426045606\\
0.0752	1.93693319301639\\
0.0753	1.93682374197807\\
0.0754	1.93671425144566\\
0.0755	1.93660472141914\\
0.0756	1.93649515189852\\
0.0757	1.9363855428838\\
0.0758	1.93627589437498\\
0.0759	1.93616620637205\\
0.076	1.93605647887502\\
0.0761	1.93594671188389\\
0.0762	1.93583690539865\\
0.0763	1.93572705941931\\
0.0764	1.93561717394588\\
0.0765	1.93550724897833\\
0.0766	1.93539728451669\\
0.0767	1.93528728056094\\
0.0768	1.93517723711109\\
0.0769	1.93506715416714\\
0.077	1.93495703172909\\
0.0771	1.93484686979693\\
0.0772	1.93473666837067\\
0.0773	1.93462642745031\\
0.0774	1.93451614703585\\
0.0775	1.93440582712728\\
0.0776	1.93429546772461\\
0.0777	1.93418506882784\\
0.0778	1.93407463043696\\
0.0779	1.93396415255199\\
0.078	1.93385363517291\\
0.0781	1.93374307829973\\
0.0782	1.93363248193244\\
0.0783	1.93352184607106\\
0.0784	1.93341117071557\\
0.0785	1.93330045586598\\
0.0786	1.93318970152228\\
0.0787	1.93307890768449\\
0.0788	1.93296807435259\\
0.0789	1.93285720152659\\
0.079	1.93274628920649\\
0.0791	1.93263533739228\\
0.0792	1.93252434608397\\
0.0793	1.93241331528156\\
0.0794	1.93230224498505\\
0.0795	1.93219113519443\\
0.0796	1.93207998590971\\
0.0797	1.93196879713089\\
0.0798	1.93185756885797\\
0.0799	1.93174630109095\\
0.08	1.93163499382982\\
0.0801	1.93152364707459\\
0.0802	1.93141226082526\\
0.0803	1.93130083508182\\
0.0804	1.93118936984428\\
0.0805	1.93107786511264\\
0.0806	1.9309663208869\\
0.0807	1.93085473716705\\
0.0808	1.93074311395311\\
0.0809	1.93063145124506\\
0.081	1.9305197490429\\
0.0811	1.93040800734665\\
0.0812	1.93029622615629\\
0.0813	1.93018440547183\\
0.0814	1.93007254529327\\
0.0815	1.92996064562061\\
0.0816	1.92984870645384\\
0.0817	1.92973672779297\\
0.0818	1.929624709638\\
0.0819	1.92951265198892\\
0.082	1.92940055484575\\
0.0821	1.92928841820847\\
0.0822	1.92917624207708\\
0.0823	1.9290640264516\\
0.0824	1.92895177133201\\
0.0825	1.92883947671832\\
0.0826	1.92872714261053\\
0.0827	1.92861476900864\\
0.0828	1.92850235591264\\
0.0829	1.92838990332254\\
0.083	1.92827741123834\\
0.0831	1.92816487966004\\
0.0832	1.92805230858763\\
0.0833	1.92793969802112\\
0.0834	1.92782704796051\\
0.0835	1.9277143584058\\
0.0836	1.92760162935698\\
0.0837	1.92748886081406\\
0.0838	1.92737605277704\\
0.0839	1.92726320524592\\
0.084	1.92715031822069\\
0.0841	1.92703739170136\\
0.0842	1.92692442568793\\
0.0843	1.9268114201804\\
0.0844	1.92669837517876\\
0.0845	1.92658529068303\\
0.0846	1.92647216669318\\
0.0847	1.92635900320924\\
0.0848	1.9262458002312\\
0.0849	1.92613255775905\\
0.085	1.9260192757928\\
0.0851	1.92590595433244\\
0.0852	1.92579259337799\\
0.0853	1.92567919292943\\
0.0854	1.92556575298677\\
0.0855	1.92545227355001\\
0.0856	1.92533875461914\\
0.0857	1.92522519619417\\
0.0858	1.9251115982751\\
0.0859	1.92499796086193\\
0.086	1.92488428395466\\
0.0861	1.92477056755328\\
0.0862	1.9246568116578\\
0.0863	1.92454301626822\\
0.0864	1.92442918138453\\
0.0865	1.92431530700674\\
0.0866	1.92420139313485\\
0.0867	1.92408743976886\\
0.0868	1.92397344690877\\
0.0869	1.92385941455457\\
0.087	1.92374534270627\\
0.0871	1.92363123136387\\
0.0872	1.92351708052736\\
0.0873	1.92340289019676\\
0.0874	1.92328866037205\\
0.0875	1.92317439105324\\
0.0876	1.92306008224032\\
0.0877	1.92294573393331\\
0.0878	1.92283134613219\\
0.0879	1.92271691883696\\
0.088	1.92260245204764\\
0.0881	1.92248794576421\\
0.0882	1.92237339998668\\
0.0883	1.92225881471505\\
0.0884	1.92214418994932\\
0.0885	1.92202952568948\\
0.0886	1.92191482193554\\
0.0887	1.9218000786875\\
0.0888	1.92168529594536\\
0.0889	1.92157047370911\\
0.089	1.92145561197876\\
0.0891	1.92134071075431\\
0.0892	1.92122577003576\\
0.0893	1.9211107898231\\
0.0894	1.92099577011634\\
0.0895	1.92088071091548\\
0.0896	1.92076561222052\\
0.0897	1.92065047403145\\
0.0898	1.92053529634829\\
0.0899	1.92042007917102\\
0.09	1.92030482249964\\
0.0901	1.92018952633417\\
0.0902	1.92007419067459\\
0.0903	1.91995881552091\\
0.0904	1.91984340087312\\
0.0905	1.91972794673124\\
0.0906	1.91961245309525\\
0.0907	1.91949691996516\\
0.0908	1.91938134734097\\
0.0909	1.91926573522267\\
0.091	1.91915008361027\\
0.0911	1.91903439250377\\
0.0912	1.91891866190317\\
0.0913	1.91880289180847\\
0.0914	1.91868708221966\\
0.0915	1.91857123313675\\
0.0916	1.91845534455974\\
0.0917	1.91833941648862\\
0.0918	1.9182234489234\\
0.0919	1.91810744186408\\
0.092	1.91799139531066\\
0.0921	1.91787530926314\\
0.0922	1.91775918372151\\
0.0923	1.91764301868578\\
0.0924	1.91752681415595\\
0.0925	1.91741057013201\\
0.0926	1.91729428661398\\
0.0927	1.91717796360184\\
0.0928	1.9170616010956\\
0.0929	1.91694519909525\\
0.093	1.9168287576008\\
0.0931	1.91671227661225\\
0.0932	1.9165957561296\\
0.0933	1.91647919615285\\
0.0934	1.91636259668199\\
0.0935	1.91624595771703\\
0.0936	1.91612927925797\\
0.0937	1.91601256130481\\
0.0938	1.91589580385754\\
0.0939	1.91577900691617\\
0.094	1.9156621704807\\
0.0941	1.91554529455113\\
0.0942	1.91542837912745\\
0.0943	1.91531142420967\\
0.0944	1.91519442979779\\
0.0945	1.91507739589181\\
0.0946	1.91496032249172\\
0.0947	1.91484320959753\\
0.0948	1.91472605720924\\
0.0949	1.91460886532685\\
0.095	1.91449163395035\\
0.0951	1.91437436307975\\
0.0952	1.91425705271505\\
0.0953	1.91413970285625\\
0.0954	1.91402231350334\\
0.0955	1.91390488465634\\
0.0956	1.91378741631523\\
0.0957	1.91366990848001\\
0.0958	1.9135523611507\\
0.0959	1.91343477432728\\
0.096	1.91331714800976\\
0.0961	1.91319948219814\\
0.0962	1.91308177689241\\
0.0963	1.91296403209258\\
0.0964	1.91284624779865\\
0.0965	1.91272842401062\\
0.0966	1.91261056072848\\
0.0967	1.91249265795225\\
0.0968	1.91237471568191\\
0.0969	1.91225673391746\\
0.097	1.91213871265892\\
0.0971	1.91202065190627\\
0.0972	1.91190255165952\\
0.0973	1.91178441191867\\
0.0974	1.91166623268371\\
0.0975	1.91154801395466\\
0.0976	1.9114297557315\\
0.0977	1.91131145801424\\
0.0978	1.91119312080287\\
0.0979	1.9110747440974\\
0.098	1.91095632789783\\
0.0981	1.91083787220416\\
0.0982	1.91071937701639\\
0.0983	1.91060084233451\\
0.0984	1.91048226815853\\
0.0985	1.91036365448845\\
0.0986	1.91024500132427\\
0.0987	1.91012630866598\\
0.0988	1.91000757651359\\
0.0989	1.9098888048671\\
0.099	1.9097699937265\\
0.0991	1.90965114309181\\
0.0992	1.90953225296301\\
0.0993	1.90941332334011\\
0.0994	1.9092943542231\\
0.0995	1.909175345612\\
0.0996	1.90905629750679\\
0.0997	1.90893720990748\\
0.0998	1.90881808281406\\
0.0999	1.90869891622655\\
0.1	1.90857971014493\\
0.1001	1.90846046456921\\
0.1002	1.90834117949938\\
0.1003	1.90822185493546\\
0.1004	1.90810249087743\\
0.1005	1.9079830873253\\
0.1006	1.90786364427906\\
0.1007	1.90774416173873\\
0.1008	1.90762463970429\\
0.1009	1.90750507817575\\
0.101	1.90738547715311\\
0.1011	1.90726583663636\\
0.1012	1.90714615662551\\
0.1013	1.90702643712056\\
0.1014	1.90690667812151\\
0.1015	1.90678687962835\\
0.1016	1.9066670416411\\
0.1017	1.90654716415974\\
0.1018	1.90642724718427\\
0.1019	1.90630729071471\\
0.102	1.90618729475104\\
0.1021	1.90606725929327\\
0.1022	1.9059471843414\\
0.1023	1.90582706989542\\
0.1024	1.90570691595535\\
0.1025	1.90558672252117\\
0.1026	1.90546648959288\\
0.1027	1.9053462171705\\
0.1028	1.90522590525401\\
0.1029	1.90510555384342\\
0.103	1.90498516293873\\
0.1031	1.90486473253993\\
0.1032	1.90474426264704\\
0.1033	1.90462375326004\\
0.1034	1.90450320437894\\
0.1035	1.90438261600373\\
0.1036	1.90426198813442\\
0.1037	1.90414132077102\\
0.1038	1.9040206139135\\
0.1039	1.90389986756189\\
0.104	1.90377908171617\\
0.1041	1.90365825637635\\
0.1042	1.90353739154243\\
0.1043	1.90341648721441\\
0.1044	1.90329554339228\\
0.1045	1.90317456007605\\
0.1046	1.90305353726572\\
0.1047	1.90293247496129\\
0.1048	1.90281137316275\\
0.1049	1.90269023187011\\
0.105	1.90256905108337\\
0.1051	1.90244783080253\\
0.1052	1.90232657102758\\
0.1053	1.90220527175853\\
0.1054	1.90208393299538\\
0.1055	1.90196255473813\\
0.1056	1.90184113698677\\
0.1057	1.90171967974131\\
0.1058	1.90159818300175\\
0.1059	1.90147664676809\\
0.106	1.90135507104032\\
0.1061	1.90123345581845\\
0.1062	1.90111180110248\\
0.1063	1.90099010689241\\
0.1064	1.90086837318823\\
0.1065	1.90074659998996\\
0.1066	1.90062478729758\\
0.1067	1.90050293511109\\
0.1068	1.90038104343051\\
0.1069	1.90025911225582\\
0.107	1.90013714158703\\
0.1071	1.90001513142414\\
0.1072	1.89989308176714\\
0.1073	1.89977099261604\\
0.1074	1.89964886397084\\
0.1075	1.89952669583154\\
0.1076	1.89940448819814\\
0.1077	1.89928224107063\\
0.1078	1.89915995444902\\
0.1079	1.89903762833331\\
0.108	1.89891526272349\\
0.1081	1.89879285761957\\
0.1082	1.89867041302155\\
0.1083	1.89854792892943\\
0.1084	1.89842540534321\\
0.1085	1.89830284226288\\
0.1086	1.89818023968845\\
0.1087	1.89805759761992\\
0.1088	1.89793491605728\\
0.1089	1.89781219500055\\
0.109	1.89768943444971\\
0.1091	1.89756663440476\\
0.1092	1.89744379486572\\
0.1093	1.89732091583257\\
0.1094	1.89719799730532\\
0.1095	1.89707503928397\\
0.1096	1.89695204176852\\
0.1097	1.89682900475896\\
0.1098	1.8967059282553\\
0.1099	1.89658281225754\\
0.11	1.89645965676568\\
0.1101	1.89633646177971\\
0.1102	1.89621322729964\\
0.1103	1.89608995332547\\
0.1104	1.8959666398572\\
0.1105	1.89584328689482\\
0.1106	1.89571989443834\\
0.1107	1.89559646248776\\
0.1108	1.89547299104308\\
0.1109	1.89534948010429\\
0.111	1.8952259296714\\
0.1111	1.89510233974441\\
0.1112	1.89497871032332\\
0.1113	1.89485504140812\\
0.1114	1.89473133299882\\
0.1115	1.89460758509542\\
0.1116	1.89448379769792\\
0.1117	1.89435997080631\\
0.1118	1.89423610442061\\
0.1119	1.8941121985408\\
0.112	1.89398825316688\\
0.1121	1.89386426829887\\
0.1122	1.89374024393675\\
0.1123	1.89361618008053\\
0.1124	1.89349207673021\\
0.1125	1.89336793388578\\
0.1126	1.89324375154725\\
0.1127	1.89311952971462\\
0.1128	1.89299526838789\\
0.1129	1.89287096756705\\
0.113	1.89274662725212\\
0.1131	1.89262224744308\\
0.1132	1.89249782813993\\
0.1133	1.89237336934269\\
0.1134	1.89224887105134\\
0.1135	1.89212433326589\\
0.1136	1.89199975598634\\
0.1137	1.89187513921269\\
0.1138	1.89175048294493\\
0.1139	1.89162578718307\\
0.114	1.89150105192711\\
0.1141	1.89137627717704\\
0.1142	1.89125146293287\\
0.1143	1.89112660919461\\
0.1144	1.89100171596223\\
0.1145	1.89087678323576\\
0.1146	1.89075181101518\\
0.1147	1.8906267993005\\
0.1148	1.89050174809172\\
0.1149	1.89037665738884\\
0.115	1.89025152719185\\
0.1151	1.89012635750076\\
0.1152	1.89000114831557\\
0.1153	1.88987589963628\\
0.1154	1.88975061146288\\
0.1155	1.88962528379538\\
0.1156	1.88949991663378\\
0.1157	1.88937450997808\\
0.1158	1.88924906382827\\
0.1159	1.88912357818436\\
0.116	1.88899805304635\\
0.1161	1.88887248841424\\
0.1162	1.88874688428802\\
0.1163	1.8886212406677\\
0.1164	1.88849555755328\\
0.1165	1.88836983494476\\
0.1166	1.88824407284213\\
0.1167	1.8881182712454\\
0.1168	1.88799243015457\\
0.1169	1.88786654956964\\
0.117	1.8877406294906\\
0.1171	1.88761466991746\\
0.1172	1.88748867085022\\
0.1173	1.88736263228888\\
0.1174	1.88723655423343\\
0.1175	1.88711043668389\\
0.1176	1.88698427964024\\
0.1177	1.88685808310248\\
0.1178	1.88673184707063\\
0.1179	1.88660557154467\\
0.118	1.88647925652461\\
0.1181	1.88635290201045\\
0.1182	1.88622650800218\\
0.1183	1.88610007449981\\
0.1184	1.88597360150334\\
0.1185	1.88584708901277\\
0.1186	1.8857205370281\\
0.1187	1.88559394554932\\
0.1188	1.88546731457644\\
0.1189	1.88534064410946\\
0.119	1.88521393414837\\
0.1191	1.88508718469318\\
0.1192	1.8849603957439\\
0.1193	1.8848335673005\\
0.1194	1.88470669936301\\
0.1195	1.88457979193141\\
0.1196	1.88445284500571\\
0.1197	1.88432585858591\\
0.1198	1.88419883267201\\
0.1199	1.884071767264\\
0.12	1.88394466236189\\
0.1201	1.88381751796568\\
0.1202	1.88369033407536\\
0.1203	1.88356311069095\\
0.1204	1.88343584781243\\
0.1205	1.88330854543981\\
0.1206	1.88318120357308\\
0.1207	1.88305382221226\\
0.1208	1.88292640135733\\
0.1209	1.88279894100829\\
0.121	1.88267144116516\\
0.1211	1.88254390182792\\
0.1212	1.88241632299659\\
0.1213	1.88228870467115\\
0.1214	1.8821610468516\\
0.1215	1.88203334953795\\
0.1216	1.88190561273021\\
0.1217	1.88177783642835\\
0.1218	1.8816500206324\\
0.1219	1.88152216534235\\
0.122	1.88139427055819\\
0.1221	1.88126633627993\\
0.1222	1.88113836250756\\
0.1223	1.8810103492411\\
0.1224	1.88088229648053\\
0.1225	1.88075420422586\\
0.1226	1.88062607247709\\
0.1227	1.88049790123421\\
0.1228	1.88036969049723\\
0.1229	1.88024144026615\\
0.123	1.88011315054097\\
0.1231	1.87998482132168\\
0.1232	1.87985645260829\\
0.1233	1.8797280444008\\
0.1234	1.87959959669921\\
0.1235	1.87947110950352\\
0.1236	1.87934258281372\\
0.1237	1.87921401662982\\
0.1238	1.87908541095182\\
0.1239	1.87895676577971\\
0.124	1.8788280811135\\
0.1241	1.87869935695319\\
0.1242	1.87857059329878\\
0.1243	1.87844179015027\\
0.1244	1.87831294750765\\
0.1245	1.87818406537093\\
0.1246	1.87805514374011\\
0.1247	1.87792618261518\\
0.1248	1.87779718199615\\
0.1249	1.87766814188303\\
0.125	1.87753906227579\\
0.1251	1.87740994317446\\
0.1252	1.87728078457902\\
0.1253	1.87715158648948\\
0.1254	1.87702234890584\\
0.1255	1.8768930718281\\
0.1256	1.87676375525625\\
0.1257	1.8766343991903\\
0.1258	1.87650500363025\\
0.1259	1.8763755685761\\
0.126	1.87624609402784\\
0.1261	1.87611657998548\\
0.1262	1.87598702644902\\
0.1263	1.87585743341845\\
0.1264	1.87572780089379\\
0.1265	1.87559812887502\\
0.1266	1.87546841736215\\
0.1267	1.87533866635517\\
0.1268	1.8752088758541\\
0.1269	1.87507904585892\\
0.127	1.87494917636964\\
0.1271	1.87481926738625\\
0.1272	1.87468931890877\\
0.1273	1.87455933093718\\
0.1274	1.87442930347149\\
0.1275	1.8742992365117\\
0.1276	1.8741691300578\\
0.1277	1.8740389841098\\
0.1278	1.8739087986677\\
0.1279	1.8737785737315\\
0.128	1.87364830930119\\
0.1281	1.87351800537678\\
0.1282	1.87338766195827\\
0.1283	1.87325727904566\\
0.1284	1.87312685663895\\
0.1285	1.87299639473813\\
0.1286	1.87286589334321\\
0.1287	1.87273535245418\\
0.1288	1.87260477207106\\
0.1289	1.87247415219383\\
0.129	1.8723434928225\\
0.1291	1.87221279395707\\
0.1292	1.87208205559753\\
0.1293	1.8719512777439\\
0.1294	1.87182046039616\\
0.1295	1.87168960355431\\
0.1296	1.87155870721837\\
0.1297	1.87142777138832\\
0.1298	1.87129679606417\\
0.1299	1.87116578124592\\
0.13	1.87103472693356\\
0.1301	1.87090363312711\\
0.1302	1.87077249982655\\
0.1303	1.87064132703188\\
0.1304	1.87051011474312\\
0.1305	1.87037886296025\\
0.1306	1.87024757168328\\
0.1307	1.87011624091221\\
0.1308	1.86998487064704\\
0.1309	1.86985346088776\\
0.131	1.86972201163438\\
0.1311	1.8695905228869\\
0.1312	1.86945899464532\\
0.1313	1.86932742690963\\
0.1314	1.86919581967984\\
0.1315	1.86906417295595\\
0.1316	1.86893248673795\\
0.1317	1.86880076102586\\
0.1318	1.86866899581966\\
0.1319	1.86853719111936\\
0.132	1.86840534692495\\
0.1321	1.86827346323645\\
0.1322	1.86814154005384\\
0.1323	1.86800957737713\\
0.1324	1.86787757520631\\
0.1325	1.8677455335414\\
0.1326	1.86761345238238\\
0.1327	1.86748133172926\\
0.1328	1.86734917158204\\
0.1329	1.86721697194071\\
0.133	1.86708473280528\\
0.1331	1.86695245417575\\
0.1332	1.86682013605212\\
0.1333	1.86668777843438\\
0.1334	1.86655538132254\\
0.1335	1.8664229447166\\
0.1336	1.86629046861656\\
0.1337	1.86615795302241\\
0.1338	1.86602539793417\\
0.1339	1.86589280335182\\
0.134	1.86576016927536\\
0.1341	1.86562749570481\\
0.1342	1.86549478264015\\
0.1343	1.86536203008139\\
0.1344	1.86522923802853\\
0.1345	1.86509640648156\\
0.1346	1.8649635354405\\
0.1347	1.86483062490532\\
0.1348	1.86469767487605\\
0.1349	1.86456468535268\\
0.135	1.8644316563352\\
0.1351	1.86429858782362\\
0.1352	1.86416547981794\\
0.1353	1.86403233231815\\
0.1354	1.86389914532427\\
0.1355	1.86376591883628\\
0.1356	1.86363265285418\\
0.1357	1.86349934737799\\
0.1358	1.86336600240769\\
0.1359	1.86323261794329\\
0.136	1.86309919398479\\
0.1361	1.86296573053219\\
0.1362	1.86283222758548\\
0.1363	1.86269868514467\\
0.1364	1.86256510320976\\
0.1365	1.86243148178074\\
0.1366	1.86229782085763\\
0.1367	1.86216412044041\\
0.1368	1.86203038052909\\
0.1369	1.86189660112366\\
0.137	1.86176278222414\\
0.1371	1.86162892383051\\
0.1372	1.86149502594278\\
0.1373	1.86136108856094\\
0.1374	1.86122711168501\\
0.1375	1.86109309531497\\
0.1376	1.86095903945083\\
0.1377	1.86082494409258\\
0.1378	1.86069080924024\\
0.1379	1.86055663489379\\
0.138	1.86042242105324\\
0.1381	1.86028816771858\\
0.1382	1.86015387488983\\
0.1383	1.86001954256697\\
0.1384	1.85988517075001\\
0.1385	1.85975075943895\\
0.1386	1.85961630863378\\
0.1387	1.85948181833451\\
0.1388	1.85934728854114\\
0.1389	1.85921271925367\\
0.139	1.85907811047209\\
0.1391	1.85894346219641\\
0.1392	1.85880877442663\\
0.1393	1.85867404716275\\
0.1394	1.85853928040476\\
0.1395	1.85840447415268\\
0.1396	1.85826962840649\\
0.1397	1.85813474316619\\
0.1398	1.8579998184318\\
0.1399	1.8578648542033\\
0.14	1.8577298504807\\
};
\addlegendentry{Quadratic Interpolation};

\end{axis}
\end{tikzpicture}% 
\caption{The plot of Height-vs-Time}

\end{figure}
\end{document}